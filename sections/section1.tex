\documentclass[../main.tex]{subfiles}

\begin{document}

\subsection{Sets}
\begin{itemize}
    \item A \textsf{set} is a collection of distinct objects with a precise description that provides a way of deciding whether given objects are in it.
    \item $x = y$: $x$ and $y$ are equal.
    \item $x \neq y$: $x$ and $y$ are not equal.
\end{itemize}

\begin{defn} \label{def:set}
    The objects in a set are its \textsf{elements} or \textsf{members}.
    When $x$ is an element of $A$, we write $x \in A$ and say $x$ belongs to $A$.
    When $x$ is not in $A$, we write $x \notin A$.
\end{defn}

We use the symbols $\mathbb N$, $\mathbb Z$, $\mathbb Q$, $\mathbb R$, and $\mathbb C$ to denote the following sets of numbers:
\begin{itemize}
    \item $\mathbb{N} = \{ 0, 1, 2, 3, \cdots \}$ is the set of natural numbers.
    \item $\mathbb{Z} = \{ \cdots, -2, -1, 0, 1, 2, \cdots \}$ is the set of integers.
    \item $\mathbb{Q} = \left\{ \frac a b \middle | a, b \in \mathbb{Z} \text{ and } b \neq 0\right\}$ is the set of rational numbers.
    \item $\mathbb R$ is the set of real numbers.
    \item $\mathbb{C} = \left\{ a + ib \middle | a, b \in \mathbb{R} \text{ and } i = \sqrt{-1}\right\}$
\end{itemize}

\begin{defn} \label{def:eq}
    Sets $A$ and $B$ are \textsf{equal}, written $A = B$, if they have the same elements.
    The \textsf{empty set}, written $\varnothing$, is the unique set with no elements.
\end{defn}

\subsection{Subsets}
\begin{defn} \label{def:sub}
    Let $A$ and $B$ be sets.
    We say that $A$ is a \textsf{subset} of $B$, written $A \subseteq B$, if all elements of $A$ are also elements of $B$.
    $B$ is a \textsf{proper subset} of a set $A$, written $A \subset B$, if $B$ is a subset of $A$ that is not $A$ itself.
\end{defn}

\begin{prop} \label{prop:sub}
    Let $A$, $B$, and $C$ be sets.
    Then the following properties hold:
    \begin{enumerate}
        \item $A \subseteq A$
        \item $A \subseteq B \wedge B \subseteq A \Rightarrow A = B$
            % \item If $A \subseteq B$ and $B \subseteq A$, then $A = B$.
        \item $A \subseteq B \wedge B \subseteq C \Rightarrow A \subseteq C$
            % \item If $A \subseteq B$ and $B \subseteq C$, then $A \subseteq C$.
    \end{enumerate}
\end{prop}
\begin{proof}
    % \text{}\\
    1. $A \subseteq A$
    \begin{align*}
        A = A &\Rightarrow (x \in A \Leftrightarrow x \in A) &\text{by Definition~\ref{def:eq}}\\
              &\Rightarrow (x \in A \Rightarrow x \in A)\\
              &\Rightarrow (A \subseteq A) & \text{by Definition~\ref{def:sub}}
    \end{align*}

    2. $A \subseteq B \wedge B \subseteq A \Rightarrow A = B$
    \begin{align*}
        (A \subseteq B \wedge B \subseteq A) &\Rightarrow [(x \in A \Rightarrow x \in B) \wedge (x \in B \Rightarrow x \in A)] &\text{by Definition~\ref{def:sub}}\\
                                             &\Rightarrow (x \in A \Leftrightarrow x \in B)\\
                                             &\Rightarrow A = B &\text{by Definition~\ref{def:eq}}
    \end{align*}

    3. $A \subseteq B \wedge B \subseteq C \Rightarrow A \subseteq C$
    \begin{align*}
        (A \subseteq B \wedge B \subseteq C) &\Rightarrow [(x \in A \Rightarrow x \in B) \wedge (x \in B \Rightarrow x \in C)] &\text{by Definition~\ref{def:sub}}\\
                                             &\Rightarrow (x \in A \Rightarrow x \in C)\\
                                             &\Rightarrow A \subseteq C &\text{by Definition~\ref{def:sub}}
    \end{align*}
\end{proof}

\begin{defn} \label{def:pow}
    The \textsf{power set} of a set $A$, denoted by $\mathcal{P} (A)$, is a set of all subsets of $A$.
\end{defn}

\begin{ex}
    Let $x \neq y$ and $A = \{ x, y\}$.
    Then,
    \[
        \mathcal{P} (A) = \{ \varnothing, \{x\}, \{y\}, \{x, y\}\}
    \]
\end{ex}

\subsection{Set Operators}
\begin{defn} \label{def:ops}
    Let $A$ and $B$ be sets.
    Their \textsf{union}, written $A \cup B$, consists of all elements in $A$ or $B$.
    Their \textsf{intersection}, written $A \cap B$, consists of all elements in both $A$ and $B$.
    Their \textsf{difference}, written $A - B$, consists of all elements of $A$ that are not in $B$.
    Two sets are \textsf{disjoint} if their intersection is an empty set $\varnothing$.
    If a set $A$ is contained in some universe $\mathcal U$ under discussion, then the complement $A^\mathsf{c}$ of $A$ is the set of elements of $\mathcal U$ not in $A$.
\end{defn}

\begin{prop} \label{prop:union}
    Let $A$ and $B$ be sets.
    Then,
    \[
        A \subseteq B \Leftrightarrow A \cup B = B
    \]
\end{prop}
\begin{proof}
    1. $A \subseteq B \Rightarrow A \cup B = B$
    \begin{align*}
        A \subseteq B &\Rightarrow (x \in A \Rightarrow x \in B) &\text{by Definition~\ref{def:sub}}\\
                      &\Rightarrow (x \in A \vee x \in B \Rightarrow x \in B)\\
                      &\Rightarrow (x \in A \cup B \Rightarrow x \in B) &\text{by Definition~\ref{def:ops}}\\
                      &\Rightarrow [(x \in A \cup B \Rightarrow x \in B) \wedge B \subseteq A \cup B] &\text{by Definition~\ref{def:ops}}\\
                      &\Rightarrow (A \cup B \subseteq B \wedge B \subseteq A \cup B) &\text{by Definition~\ref{def:sub}}\\
                      &\Rightarrow A \cup B = B &\text{by Proposition~\ref{prop:sub}}
    \end{align*}
    2. $A \subseteq B \Leftarrow A \cup B = B$
    \begin{align*}
        A \cup B = B &\Rightarrow A \cup B \subseteq B &\text{by Proposition~\ref{prop:sub}}\\
                     &\Rightarrow (A \cup B \subseteq B \wedge A \subseteq A \cup B) &\text{by Definition~\ref{def:ops}}\\
                     &\Rightarrow A \subseteq B &\text{by Proposition~\ref{prop:sub}}
    \end{align*}
\end{proof}

\begin{prop} \label{prop:union_more}
    Let $A$, $B$, and $C$ be sets.
    Then
    \begin{enumerate}
        \item $A \cup A = A$
        \item $A \cup B = B \cup A$
        \item $(A \cup B) \cup C = A \cup (B \cup C)$
    \end{enumerate}
\end{prop}
\begin{proof}
    1. $A \cup A = A$
    \begin{align*}
        x \in A \cup A &\Leftrightarrow x \in A \vee x \in A &\text{by Definition~\ref{def:ops}}\\
                       &\Leftrightarrow x \in A
    \end{align*}
    Therefore, $A \cup A = A$.\\
    2. $A \cup B = B \cup A$
    \begin{align*}
        x \in A \cup B &\Leftrightarrow (x \in A \vee x \in B) &\text{by Definition~\ref{def:ops}}\\
                       &\Leftrightarrow (x \in B \vee x \in A)\\
                       &\Leftrightarrow x \in B \cup A &\text{by Definition~\ref{def:ops}}
    \end{align*}
    Therefore, $A \cup B = B \cup A$.\\
    3. $(A \cup B) \cup C = A \cup (B \cup C)$
    \begin{align*}
        x \in (A \cup B) \cup C &\Leftrightarrow x \in A \cup B \vee x \in C &\text{by Definition~\ref{def:ops}}\\
                                &\Leftrightarrow (x \in A \vee x \in B) \vee x \in C &\text{by Definition~\ref{def:ops}}\\
                                &\Leftrightarrow x \in A \vee (x \in B \vee x \in C)\\
                                &\Leftrightarrow x \in A \vee x \in B \cup C\\
                                &\Leftrightarrow x \in A \cup (B \cup C)
    \end{align*}
\end{proof}

\begin{ex} \label{ex:int_sub}
    Let $A$ and $B$ be sets.
    Show that
    \[
        A \subseteq B \Leftrightarrow A \cap B = A.
    \]
\end{ex}
\begin{proof}
    In the same way as in Proposition~\ref{prop:union}.
\end{proof}

\begin{ex} \label{ex:int_ass}
    Let $A$, $B$, and $C$ be sets.
    Then prove the following properties.
    \begin{enumerate}
        \item $A \cap A = A$
        \item $A \cap B = B \cap A$
        \item $(A \cap B) \cap C = A \cap (B \cap C)$
    \end{enumerate}
\end{ex}
\begin{proof}
    In the same way as in Proposition~\ref{prop:union_more}.
\end{proof}

\begin{prop}
    Let $A$, $B$, $C$ be sets. Then
    \begin{enumerate}
        \item $A \cup (A \cap B) = A$
        \item $A \cap (A \cup B) = A$
        \item $A \cup (B \cap C) = (A \cup B) \cap (A \cup C)$
        \item$A \cap (B \cup C) = (A \cap B) \cup (A \cap C)$
    \end{enumerate}
\end{prop}
\begin{proof}
    1. $A \cup (A \cap B) = A$
    \begin{align*}
        x \in A \cap B &\Rightarrow (x \in A \wedge x \in B) &\text{by Definition~\ref{def:ops}}\\
                       &\Rightarrow x \in A
    \end{align*}
    Thus, $A \cap B \subseteq A$.
    Therefore, by Proposition~\ref{prop:union}, $A \cup (A \cap B) = A$.

    2. $A \cap (A \cup B) = A$
    \begin{align*}
        x \in A &\Rightarrow (x \in A \vee x \in B) \\
                &\Rightarrow x \in A \cup B &\text{by Definition~\ref{def:ops}}\
    \end{align*}
    Thus, $A \subseteq A \cup B$.
    Therefore, by Example~\ref{ex:int_sub}, $A \cap (A \cup B) = A$.

    3. $A \cup (B \cap C) = (A \cup B) \cap (A \cup C)$
    \begin{align*}
        x \in A \cup (B \cap C) &\Leftrightarrow (x \in A \vee x \in B \cap C) &\text{by Definition~\ref{def:ops}}\\
                                &\Leftrightarrow [x \in A \vee (x \in B \wedge x \in C)]\\
                                &\Leftrightarrow [(x \in A \vee x \in B) \wedge (x \in A \vee x \in C)]\\
                                &\Leftrightarrow (x \in A \cup B \wedge x \in A \cup C) &\text{by Definition~\ref{def:ops}}\\
                                &\Leftrightarrow x \in (A\cup B) \cap (A \cup C) &\text{by Definition~\ref{def:ops}}
    \end{align*}
    Therefore, $A \cup (B \cap C) = (A \cup B) \cap (A \cup C)$.

    4. $A \cap (B \cup C) = (A \cap B) \cup (A \cap C)$
    \begin{align*}
        x \in A \cap (B \cup C) &\Leftrightarrow (x \in A \wedge x \in B \cup C) &\text{by Definition~\ref{def:ops}}\\
                                &\Leftrightarrow [x \in A \wedge (x \in B \vee x \in C)] &\text{by Definition~\ref{def:ops}}\\
                                &\Leftrightarrow [(x \in A \wedge x \in B) \vee (x \in A \vee x \in C)]\\
                                &\Leftrightarrow (x \in A \cup B) \vee (x \in A \cup C) &\text{by Definition~\ref{def:ops}}\\
                                &\Leftrightarrow x \in (A \cap B) \cup (A \cap C) &\text{by Definition~\ref{def:ops}}
    \end{align*}
    Therefore, $A \cap (B \cup C) = (A \cap B) \cup (A \cap C)$.
\end{proof}

\begin{ex}
    Let $A$ and $B$ be sets.
    Show that
    \[
        A \subseteq B \Leftrightarrow A - B = \varnothing.
    \]
\end{ex}
\begin{proof}
    We first show that $A \subseteq B \Rightarrow A - B = \varnothing$.
    \begin{align*}
        A \subseteq B &\Rightarrow (x \in A \Rightarrow x \in B) &\text{by Definition~\ref{def:sub}}\\
                      &\Rightarrow [(x \in A \wedge x \notin B) \Rightarrow (x \in B \wedge x \notin B)]\\
                      &\Rightarrow [(x \in A \wedge x \notin B) \Rightarrow x \in \varnothing]\\
                      &\Rightarrow (x \in A - B \Rightarrow x \in \varnothing) &\text{by Definition~\ref{def:ops}}\\
                      &\Rightarrow A - B \subseteq \varnothing &\text{by Definition~\ref{def:sub}} \\
                      &\Rightarrow A - B \subseteq \varnothing \wedge \varnothing \subseteq A - B\\
                      &\Rightarrow A - B = \varnothing &\text{by Proposition~\ref{prop:sub}}
    \end{align*}
    We now show that $A - B = \varnothing \Rightarrow A \subseteq B$.
    \begin{align*}
        A - B = \varnothing &\Rightarrow A - B \subseteq \varnothing &\text{by Proposition~\ref{prop:sub}}\\
                            &\Rightarrow (x \in A - B \Rightarrow x \in \varnothing)\\
                            &\Rightarrow [(x \in A \wedge x \notin B) \Rightarrow x \in \varnothing] &\text{by Definition~\ref{def:ops}}\\
                            &\Rightarrow [(x \in A \wedge x \notin B) \vee x \in B \Rightarrow x \in \varnothing \vee x \in B]\\
                            &\Rightarrow [(x \in A \vee x \in B) \wedge (x \notin B \vee x \in B) \Rightarrow x \in \varnothing \cup B \wedge \varnothing \subseteq B] &\text{by Definition~\ref{def:ops}}\\
                            &\Rightarrow (x \in A \cup B \wedge x \in \mathcal{U} \Rightarrow x \in B) &\text{by Def.~\ref{def:ops} and Prop.~\ref{prop:union}}\\
                            &\Rightarrow (x \in A \cup B \Rightarrow x \in B)\\
                            &\Rightarrow A \cup B \subseteq B &\text{by Definition~\ref{def:sub}}\\
                            &\Rightarrow A \cup B \subseteq B \wedge B \subseteq A \cup B &\text{by Definition~\ref{def:ops}}\\
                            &\Rightarrow A \cup B = B &\text{by Proposition~\ref{prop:union}}
    \end{align*}
\end{proof}

\begin{ex}
    Let $A$, $B$,and $C$ be sets.
    Prove the following properties:
    \begin{enumerate}
        \item $A - (B \cup C) = (A - B) \cap (A - C)$
        \item $A - (B \cap C) = (A - B) \cup (A - C)$
    \end{enumerate}
\end{ex}
\begin{proof}
    Let's first prove that $A - B = A \cap B^\mathsf{c}$.
    \begin{align*}
        x \in A - B &\Leftrightarrow x \in A \wedge x \notin B &\text{by Definition~\ref{def:ops}}\\
                    &\Leftrightarrow x \in A \wedge x \in B^\mathsf{c} &\text{by Definition~\ref{def:ops}}\\
                    &\Leftrightarrow x \in A \cap B^\mathsf{c}
    \end{align*}
    Thus, $A - B = A \cap B^\mathsf{c}$. Now we prove the following:
    \begin{align*}
        (A \cup B)^\mathsf{c} &= A^\mathsf{c} \cap B^\mathsf{c}\\
        (A \cap B)^\mathsf{c} &= A^\mathsf{c} \cup B^\mathsf{c}.
    \end{align*}
    First show that $(A \cup B)^\mathsf{c} = A^\mathsf{c} \cap B^\mathsf{c}$.
    \begin{align*}
        x \in (A \cup B)^\mathsf{c} &\Leftrightarrow \neg (x \in A \cup B) &\text{by Definition~\ref{def:ops}}\\
                                    &\Leftrightarrow \neg (x \in A \vee x \
    \in B) &\text{by Definition~\ref{def:ops}}\\
           &\Leftrightarrow (\neg x \in A) \wedge (\neg x \in B)\\
           &\Leftrightarrow x \in A^\mathsf{c} \wedge x \in B^\mathsf{c} &\text{by Definition~\ref{def:ops}}\\
           &\Leftrightarrow x \in A^\mathsf{c} \cap B^\mathsf{c} &\text{by Definition~\ref{def:ops}}
    \end{align*}
    Showing $(A \cap B)^\mathsf{c} = A^\mathsf{c} \cup B^\mathsf{c}$ is analogous.

    1. $A - (B \cup C) = (A - B) \cap (A - C)$
    \begin{align*}
        A - (B \cup C) &= A \cap (B \cup C)^\mathsf{c}\\
                       &= A \cap (B^\mathsf{c} \cap C^\mathsf{c}) &\text{by Definition~\ref{def:ops}}\\
                       &= (A \cap B^\mathsf{c}) \cap (A \cap C^\mathsf{c}) &\text{by Example~\ref{ex:int_ass}}\\
                       &= (A - B) \cap (A - C)
    \end{align*}

    2. $A - (B \cap C) = (A - B) \cup (A - C)$

    It is analogous to the above.
\end{proof}

\begin{defn}
    A \textsf{list} with entries in $A$ consists of elements of $A$ in a specified order, with repetition allowed.
    A $k$-tuple is a list with $k$ entries.
    We write $A^k$ for the set of $k$-tuples with entries in $A$.
    An \textsf{ordered pair} is a list with two entries.
    The \textsf{Cartesian product} of sets $S$ and $T$, written $S \times T$, is the set $\{ (x,y) | x \in S, y \in T\}$.
    Two ordered pairs $(a, b)$ and $(c, d)$ are \textsf{equal}, written $(a, b) = (c, d)$, if $a = c$ and $b = d$.
\end{defn}
\end{document}
