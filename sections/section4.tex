\documentclass[../main.tex]{subfiles}
\begin{document}
\subsection{The Principle of Induction}
\begin{defn} \label{def:n}
    The set $\mathbb N$ of \textsf{natural numbers} is the intersection of all sets $S \subset \mathbb R$ that have following properties:
    \begin{enumerate}
        \item $1 \in S$
        \item $x \in S \Rightarrow x + 1 \in S$.
    \end{enumerate}
\end{defn}

\begin{thm}[Principle of Induction] \label{thm:induction}
    For each natural number $n$, let $P(n)$ be a mathematical statement.
    Then for each $n \in \mathbb N$ the statement $P(n)$ is true, if the following properties hold:
    \begin{enumerate}
        \item $P(1)$ is true.
        \item For $k \in \mathbb N$, if $P(k)$ is true, then $P(k+1)$ is true.
    \end{enumerate}
\end{thm}

Proving property 1 of Theorem~\ref{thm:induction} is the \textsf{basis step}, and proving property 2 is the \textsf{induction step}.

\begin{prop}
    For $n \in \mathbb N$, the formula $1 + 2 + \dots + n = \frac{n(n + 1)}{2}$ holds.
\end{prop}

\begin{prop}
    If $n \in \mathbb N$ and $q \geq 2$, then $n < q^n$.
\end{prop}
\begin{proof}
    We use induction on $n$.
    For the case $n = 1$, $1 < 2 \leq q$, so the statement holds.

    Suppose now as an induction hypothesis that $k < q^k$ for some $k \in \mathbb N$ where $q \geq 2$.
    \begin{align*}
        k + 1 < 2k &\Rightarrow k + 1 < qk\\
                   &\Rightarrow k + 1 < q \cdot q^k\\
                   &\Rightarrow k + 1 < q^{k + 1}
    \end{align*}
    Hence, by the induction principle (Theorem~\ref{thm:induction}), the given statement is true for all $n \in \mathbb N$.
\end{proof}

\begin{prop} \label{prop:prod}
    If $x_1, x_2, \dots, x_n$ are numbers in the interval $[0, 1]$, then
    \[
        \prod_{i = 1}^{n} (1 - x_i) \geq 1 - \sum_{i = 1}^n x_i.
    \]
\end{prop}
\begin{proof}
    We use induction on $n$.
    For the case $n = 1$, the given inequality holds since $1 - x_1 \geq 1- x_1$.

    Suppose now as an induction hypothesis that $\prod_{i = 1}^k (1 - x_i) \geq 1 - \sum_{i = 1}^k x_i$ for some $k \in \mathbb N$ where $k \geq 2$.
    Let $P = \prod_{i = 1}^k (1 - x_i)$ and $S = 1 - \sum_{i = 1}^k x_i$.
    Since $P$ is a product of numbers in $[0, 1]$, $P \in [0, 1]$.
    Choose any $x_{k+1}$ from $[0, 1]$.
    \begin{align*}
        P \leq 1 &\Rightarrow x_{k+1} P \leq x_{k+1}\\
                 &\Rightarrow -x_{k+1} P \geq -x_{k+1}
    \end{align*}
    From the induction hypothesis $P \geq S$,
    \begin{align*}
        -x_{k+1} P \geq -x_{k+1} \wedge P \geq S &\Rightarrow P - x_{k+1} P \geq S - x_{k+1}\\
                                                 &\Rightarrow P(1 - x_{k+1}) \geq S - x_{k+1}\\
                                                 &\Rightarrow \prod_{i = 1}^{k+1} x_i \geq 1 - \sum_{i = 1}^{k+1} x_i.
    \end{align*}
    Hence, by the induction principle, the given inequality is true for all $n \in \mathbb N$.
\end{proof}

\begin{cor}
    If $0 \leq a \leq 1$ and $n \in \mathbb N$, then $(1 - a)^n \geq 1 - na$.
\end{cor}
\begin{proof}
    This is the case when $x_1 = x_2 = \dots = x_n = a$ in Theorem~\ref{prop:prod}.
\end{proof}

\begin{prop} \label{prop:sq_pow}
    If $n \in \mathbb N$ and $n \geq 4$, then $n^2 \leq 2^n$.
\end{prop}
\begin{proof}
    We use induction on $n$.
    For the case $n = 4$, the given inequality holds since $4^2 = 16 \leq 2^4 = 16$.

    Suppose now as an induction hypothesis that $k^2 \leq 2^k$ for some $k \in \mathbb N$ where $k \geq 5$.
    \begin{align*}
        k \geq 5 &\Rightarrow k - 1 \geq 4\\
                 &\Rightarrow (k - 1)^2 \geq 16\\
                 &\Rightarrow k^2 - 2k + 1 \geq 16\\
                 &\Rightarrow k^2 - 2k - 1 \geq 14\\
                 &\Rightarrow k^2 - 2k - 1 \geq 0\\
                 &\Rightarrow k^2 \geq 2k + 1\\
                 &\Rightarrow 2k^2 \geq k^2 + 2k + 1\\
                 &\Rightarrow 2k^2 \geq (k + 1)^2
    \end{align*}
    Since $k^2 \leq 2^k$ from the induction hypothesis, we have the following:
    \begin{align*}
        k^2 \leq 2^k \wedge 2k^2 \geq (k + 1)^2 &\Rightarrow 2 \cdot 2^k \geq (k + 1)^2\\
                                                &\Rightarrow 2^{k + 1} \geq (k + 1)^2.
    \end{align*}
    Hence, by the induction principle, the given inequality is true for all $n \in \mathbb N\backslash \{1, 2, 3\}$.
\end{proof}

\subsection{Applications}
\begin{lem} \label{lem:factor}
    If $f$ is a polynomial of degree $d$, then $a$ is a zero of $f$ iff $f(x) = (x - a) h(x)$ for some polynomial $h$ of degree $d - 1$.
\end{lem}
\begin{proof}
    We use strong induction on $d$.
    For the case $d = 1$, $f(x) = c_1 x + c_0 = \left(x - \frac{c_0}{c_1}\right) \cdot c_1$, and as $x = \frac{c_0}{c_1}$ is its zero, the given statement holds.

    Suppose now as an induction hypothesis that a polynomial $f$ with a degree $i$, $a$ is a zero of $f$ iff $f(x) = (x-a) h(x)$ for some polynomial $h$ of degree $i-1$ for all positive integer $i$ less or equal to $k \geq 2$.
    Now consider a polynomial $g$ with a degree $k+1$ which has $a$ as its zero.
    We can represent $g(x) = \sum_{i = 0}^{k+1} b_i x^i$.
    Then the following holds:
    \begin{align*}
        g(x) = \sum_{i = 0}^{k+1} b_i x^i &\Leftrightarrow g(x) - g(a) = \sum_{i=0}^{k+1} b_i x^i -  \sum_{i=0}^{k+1} b_i a^i\\
                                          &\Leftrightarrow g(x) - 0 = \sum^{k+1}_{i=0} b_i (x^i - a^i).
    \end{align*}
    Since $x^i - a^i$ is a polynomial of degree less or equal to $k$, the induction hypothesis holds.
    $x = a$ is a zero, it can be factorized as $x^i - a^i = (x-a) h_i(x)$ where $h_i$ is a polynomial of degree $i-1$.
    Thus we have
    \begin{align*}
        g(x) = \sum^{k+1}_{i=0} b_i (x - a)h_i(x) &\Leftrightarrow g(x) = (x - a) \sum^{k+1}_{i=0} h_i(x)\\
                                                  &\Leftrightarrow g(x) = (x - a) r(x)
    \end{align*}
    where $r$ is a polynomial of degree $(k+1) - 1 = k$.
    Hence the given statement holds for all $d \in \mathbb N$ by the induction principle.
\end{proof}

\begin{thm}
    Every polynomial of degree $d$ has at most $d$ zeros.
\end{thm}
\begin{proof}
    For the sake of contradiction, suppose there are more than $d$ zeros for a polynomial of degree $d$.
    Let the zeros be $a_1, a_2, \dots, a_d, a_{d+1}, \dots$.
    From Lemma~\ref{lem:factor}, $f(x) = (x - a_1)(x - a_2) \dots (x - a_d) (x - a_{d+1}) h(x)$.
    However, the right hand side results in a polynomial of degree greater than or equal to $d+1$. $\lightning$

    Therefore, every polynomial of degree $d$ has at most $d$ zeros.
\end{proof}

\begin{prob}[The Handshake Problem]
    Consider $n$ married couples at a party.
    Suppose that no person shakes hands with his or her spouse, and total $2n - 1$ people other than the host shake hands with different numbers of people.
    With how many people does the hostess shake hands?
\end{prob}
\begin{sol}
    We want to show that the spouse of the one with the maximum handshakes has the minimum handshake.
    $2n - 1$ people should all have distinct numbers of handshakes, and the only possible numbers of handshakes are $0, 1, \dots, 2n - 2$.
    The one with two handshakes--the maximum handshakes--must shake hand with everyone except his/her spouse.
    Now, everyone except his/her spouse has at least one handshake.
    Therefore, the one with no handshake must be the spouse of the one with the maximum handshakes.

    We now show that the number of the handshakes the hostess has is $n - 1$.
    We use strong induction on $n$.
    For the case $n = 2$, it is trivial that the hostess has one handshake, which follows from the argument above.

    Suppose now as an induction hypothesis that the hostness has $i - 1$ handshakes when there are $i$ couples for all positive integer $i \leq k$ where $k \geq 2$.
    Now consider the case of $k + 1$ couples.
    The one with the maximum handshake, $A$, must have handshakes with everyone except his/her spouse $A^*$, and his/her spouse has no handshake as shown above.
    Remove $A$ and $A^*$.
    There are $k$ couples, all with one handshake removed attributed from $A$.
    Removing $A^*$ does not affect other people's number of handshakes.
    This is exactly the case of $k$ couples, where the hostess has $k - 1$ handshakes.
    She previously had one more handshake from $A$.
    Thus, the hostess has $k = (k + 1) - 1$ handshakes in the case of $k + 1$ couples.
    Therefore, the hostess has $n - 1$ handshakes when there are $n$ couples for all $n \in \mathbb N$ by the induction principle.
\end{sol}

\begin{prob} [The L-Tiling Problem]
    There are a large number of L-shaped tiles as illustrated on page 31 of the textbook.
    Is it possible to form the large similar region on the right with non-overlapping copies of this tile?
\end{prob}
\begin{sol}
    It is possible.
    Classify the large region by the length of its edge in modulo 3.
    It is simple to come up with a band that surrounds the edge of the region with a thickness of two.
    Recursively construct the region.
\end{sol}

\subsection{Strong Induction}
Sometimes a proof of $P(k)$ in the induction step needs the hypothesis that $P(i)$ is true for all $i < k$ (which is already used above).
By assuming more in the induction hypothesis, we make the condition statement in the induction step weaker.
Nevertheless, this weaker implication suffices to complete the proof, so we call the method \textsf{strong induction}.

\begin{thm} [Strong Induction Principle] \label{thm:stronginduction}
    Let $\{P(n) | n \in \mathbb{N} \}$ be a sequence of mathematical statements.
    If properties 1 and 2 below hold, the for every $n \in \mathbb N$, $P(n)$ is true.
    \begin{enumerate}
        \item $P(1)$ is true.
        \item For $k \geq 2$, if $P(i)$ is true for all $i < k$, then $P(k)$ is true.
    \end{enumerate}
\end{thm}
\begin{proof}
    Let the statement $Q(n)$ be ``$P(i)$ hold for all $i \leq n$.''
    From property 1, $Q(1)$ is true.
    Then, if the property 2 holds, it can be said that if $Q(k-1)$ holds, then $Q(k)$ holds.
    From the principle of induction, $Q(n)$ is true for all $n \in \mathbb N$.
    Therefore, $P(n)$ is true for all $n \in \mathbb N$.
\end{proof}

\begin{prob}
    Suppose that $n$ coins are arranged in a row.
    We remove head-up coins, one by one.
    Each time we remove a coin, we must flip the coins still present in the (at most) two positions surrounding it.
    For which arrangements of heads and tails can we remove all the coin?
    For example, $THTHT$ fails, but $THHHT$ succeeds.
    Using $\circ$ to denote gaps due to removed coins, we remove $THHHT$ via
    \[
        THHHT \rightarrow H \circ THT \rightarrow \circ \circ THT \rightarrow \circ \circ H \circ H \rightarrow \circ \circ \circ \circ H \rightarrow \circ \circ \circ \circ \circ.
    \]
\end{prob}
\begin{sol}
    From observation, we can find out an arrangement with odd numbers of head-up coins can always be removed, whereas the one the even numbers of head-up coins cannot be removed.
    We show that the observation above is true.
    We use strong induction on $n$, where $n$ is the length of an arrangement.
    Case $n=1$ is trivially true.

    Suppose as an induction hypothesis that all arrangements of length $k$ or smaller is removable iff there are odd numbers of $H$'s for some $k \in \mathbb N$.
    Consider an arrangement of length $k + 1$.\\
    \noindent \textbf{Case 1}: There are no $H$'s in the middle of the arrangement.\\
    \indent The arrangement must be either $HT\dots TT$, $TT\dots TH$, or $HT \dots TH$.
    Only consider $HT\dots TT$ and $HT \dots TH$ without loss of generality.

    \noindent \textbf{Case 1-1}: $HT\dots TT$\\
    \indent Only choice is to remove the $H$ at the start of the arrangement.
    Removing it results in $\circ HT\dots TT$ with a $k$-long arrangement not regarding $\circ$.
    From the induction hypothesis, it is removable.

    \noindent \textbf{Case 1-2}: $HT\dots TH$\\
    \indent Removing either the first or the last $H$ results in the identical type--$\circ HT\-\dots\-TH$ or $HT\-\dots\-TH\circ$.
    From the induction hypothesis, it is unremovable.
    \\
    \\
    \noindent \textbf{Case 2}: There are $H$'s in the middle of the arrangement.\\
    \noindent \textbf{Case 2-1}: There is an odd number of $H$'s\\
    \indent Find the first $H$ that occurs in an arrangement.
    There are two possible cases:\\
    \noindent \textbf{Case 2-1-1}: $T\dots THT\dots$\\
    \indent When we remove the first $H$, the arrangement becomes $T\dots T H \circ H \dots$, where the first chunk contains 1 (odd) $H$ and the latter chunk contains odd numbers of $H$'s.
    From the induction hypothesis, the two chunks are removable.\\
    \noindent \textbf{Case 2-1-1}: $T\dots THH\dots$\\
    \indent When we remove the first $H$, the arrangement becomes $T \dots TH \circ T \dots$.
    Again, both chunks contains odd numbers of $H$'s.
    From the induction hypothesis, the two chunks are removable.\\

    \noindent \textbf{Case 2-2}: There is an even numbers of $H$'s\\
    \indent There are three possible scenarios to remove $H$'s:\\
    \noindent \textbf{Case 2-2-1}: Remove $H$ in between $H$'s\\
    \indent The two chunks each contains even and odd numbers of $H$'s, which makes it impossible to remove.\\
    \noindent \textbf{Case 2-2-2}: Remove $H$ after $T$, but not between $T$'s\\
    \indent It is impossible to remove.\\
    \noindent \textbf{Case 2-2-2}: Remove $H$ between $T$'s\\
    \indent It is impossible to remove.

    Therefore, by the induction principle, an arrangement is removable iff it contains an odd number of $H$'s.
\end{sol}

\begin{prop} [Well-Ordering Property]
    Every nonempty subset of $\mathbb N$ has a least element.
\end{prop}
\begin{proof}
    The given statement is equivalent to ``Every subset of $\mathbb N$ containing any element $n \in \mathbb N$ has a least element.''
    We show this using strong induction on $n$.
    The case $n = 1$ holds as 1 is the least element of $\mathbb N$.

    Suppose now as an induction hypothesis that a subset with an element $i$ has a least element for all $i \leq k$ such that $k \geq 2$.
    Consider a subset with an element $k + 1$.
    If the subset has a smaller element than $k + 1$, it must contain at least one of $1, 2, \dots, k$.
    From the induction hypothesis, the subset has a least element.
    If not, $k + 1$ is the least element.
    Hence, from the induction principle, the given statement is true.
\end{proof}

Suppose that $S \subset \mathbb N$.
By the well-ordering property, $S^\mathsf{c}$ has a least element.
Thus when $P(n)$ fails for some $n \in \mathbb N$, there is a least $n$ where it fails.
This yields yet another approach to induction, called the \textsf{method of decent}.
We can prove $P(n)$ for all $n \in \mathbb N$ by proving that there is no least $n$ where $P(n)$ fails.
To do this, we suppose that $P(n)$ fails for some $n$ and show that $P(k)$ must fail for some $k$ less than $n$.
The existence of $k$ implies that $n > 1$, and thus we have proved the contrapositive of property 2 from Theorem~\ref{thm:stronginduction}.

\begin{thm}
    $\sqrt 2$ is irrational.
\end{thm}
\begin{proof}
    The proof using an argument, ``Assume $\sqrt 2 = \frac p q$ where $\frac p q$ is a reduced form,'' and showing that it contradicts the fact that $p$ and $q$ are relatively prime is well known.
    We approach this in a slightly different manner, using the method of decent introduced above.

    For the sake of contradiction, suppose $\sqrt 2 = \frac m n$ where $m, n \in \mathbb{N}$.
    The fraction needs not necessarily be reduced.
    We see that $2 = \frac{m^2}{n^2}$, so $2n^2 = m^2$.
    Then,
    \begin{align*}
        \frac m n &= \frac{m(m - n)}{n(m - n)}\\
                  &= \frac{m^2 - mn}{mn - n^2}\\
                  &= \frac{2n^2 - mn}{mn - n^2}\\
                  &= \frac{2n - m}{m - n}
    \end{align*}
    Since $1 < \sqrt 2 = \frac m n < 2$, $n < m < 2n$.
    Thus $0 < m - n < n$.
    Now consider a set $S = \left\{ n \middle| \sqrt 2 = \frac m n \right\} \subset \mathbb N$.
    We just showed that if $n \in S$, $m-n < n$ is also an element of $S$.
    This contradicts the well-ordering principle. $\lightning$

    Therefore, our assumption that $\sqrt 2$ can be represented as $\frac m n$ is wrong, leading to the fact that $\sqrt 2 \notin \mathbb Q$.
    From Example~\ref{ex:exists_sq2}, $\sqrt 2 \in \mathbb R$, so $\sqrt 2$ is irrational.
\end{proof}

\begin{prop} \label{prop:oddpower}
    Any $n \in \mathbb N$ can be expressed in exactly one way as the product of an odd number and a power of 2.
\end{prop}
\begin{proof}
    The given statement can be rewritten:
    \[
        (\forall n \in \mathbb{N})(\exists ! k, m \in \mathbb N)\ n = 2^k (2m + 1).
    \]

    If $n$ is odd, it immediately follows that $k = 0$ and $m = \frac{n - 1}{2}$.
    Consider the case $n$ is even.
    Define a set $S = \left\{ a \middle | a = \frac{n}{2^k} \wedge a, k \in \mathbb{N}\right\}$.
    Then $S \subset N$ from the definition.
    From the well-ordering principle, the exists a least element $\alpha$ of $S$.
    Let $l$ be a natural number such that $\frac{n}{2^l} = \alpha$.
    Since $\alpha$ is the least element of $S$, $l$ is the maximum natural number such that $n$ is divisible by $2^l$.
    The remainder of $n \div 2^l$ is thus an odd number, i.e. $2m + 1$.
    Therefore, any even number can be written in the form of $2^k (2m + 1)$.

    We now show the uniqueness of such expression.
    Suppose $n = 2^k (2m + 1) = 2^{k'} (2m' + 1)$.
    When $k = k'$, it immediately follows that $m = m'$.
    When $k \neq k'$, let $k > k'$ without loss of generality.
    Then $2^{k - k'} (2m + 1) = 2m' + 1$.
    Since $k - k' > 0$, $2^{k - k'}$ is even, whereas $2m' + 1$ is odd. $\lightning$
    Therefore, such expression is unique.
\end{proof}

\begin{prob} [Sums of Consecutive Positive Integers]
    Prove that a natural number $n$ is a sum of consecutive smaller natural numbers iff $n$ is not a power of 2.
\end{prob}
\begin{proof}
    We first show that if $n$ is a sum of consecutive smaller natural numbers, $n$ is not a power of 2.
    Let $n = a + (a+1) + \dots + (a+k) = \frac{(2a + k)(k+1)}{2}$.
    If $k$ is even, $k+1$ is not.
    If $k$ is odd, $2a + k$ is not.
    Therefore, we see that $(2a + k)(k + 1)$ is not a power of 2.

    We now show that if $n$ is not a power of 2, it is a sum of consecutive smaller natural numbers.
    From Proposition~\ref{prop:oddpower}, $n = 2^\alpha (2 \beta + 1)$, where $\beta \neq 0$.
    There are two cases:\\
    \noindent \textbf{Case 1}: $2^\alpha \geq \beta \geq 1$\\
    \indent $2^\alpha - \beta \geq 0$, so
    \begin{align*}
        n &= 2^\alpha (2\beta + 1)\\
          &= \frac{\left[2 \left(2^\alpha - \beta\right) + 2 \beta\right](2\beta + 1)}{2}\\
          &= \left(2^\alpha - \beta\right) + \left(2^\alpha - \beta + 1\right) + \dots + \left(2^\alpha + \beta\right).
    \end{align*}

    \noindent \textbf{Case 2}: $\beta \geq 2^\alpha \geq 1$\\
    \indent $\beta - 2^\alpha \geq 0$, so $\beta - 2^\alpha + 1 \geq 1$.
    Then,
    \begin{align*}
        n &= 2^\alpha (2\beta + 1)\\
          &= \frac{\left[2\left(\beta - 2^\alpha + 1\right) + \left(2^{\alpha+1} - 1\right)\right]\left[\left(2^{\alpha + 1} - 1\right) + 1\right]}{2}\\
          &= \left(\beta - 2^\alpha + 1\right) + \left(\beta - 2^\alpha + 2\right) + \dots + \left(\beta + 2^\alpha\right).
    \end{align*}

    Therefore, the given statement holds.
\end{proof}

\subsection{Exercises for Chapter 4}
\begin{exercise}
    Prove that $n^3 + 20 > n^2 + 15n$ for all $n \in \mathbb N$.
\end{exercise}
\begin{proof}
    We first show that $4k^2 + 18k + 1 \geq (k + 1)^2 + 15(k + 1)$ for $k \geq 3$.
    Simplifying the inequality yields $\left( k + \frac 1 6 \right)^2 - \left(5 + \frac{1}{36}\right) \geq 0$.
    The inequality is true for all $k \geq 3$.

    We use induction on $n$.
    Substituting $n = 1, 2$ shows that the inequality given in the problem holds.
    Suppose now as an induction hypothesis that $k^3 + 20 > k^2 + 15k$ for some $k \geq 3$.
    Then the following holds.
    \begin{align*}
        (k + 1)^3 + 20 &= k^3 + 3k^2 + 3k + 1 + 20\\
                       &= (k^3 + 20) + (3k^2 + 3k + 1)\\
                       &> (k^2 + 15k) + (3k^2 + 3k + 1) &\text{From the induction hyp.}\\
                       &= 4k^2 + 18k + 1\\
                       &\geq (k + 1)^2 + 15(k + 1) &\text{As shown above}
    \end{align*}
    Hence, by the induction principle, the given inequality holds for any $n \in \mathbb N$.
\end{proof}

\begin{exercise}
    For $n \in \mathbb N$, when does $3^n > n^4$ hold?
\end{exercise}
\begin{sol}
    It can be checked that $n = 1$ satisfies the given inequality, but $n = 2, 3, 4, 5, 6, 7$ do not by substituting the values.
    We will show that $n \geq 8$ satisfies the given inequality using induction on $n$.
    $3^8 = 65651 > 8^4 = 4096$, so $n = 8$ satisfies the given inequality.

    Suppose now as an induction hypothesis that $3^k > k^4$ for some $k \geq 9$.
    We need to show that the following two inequalities hold where $k \geq 9$::
    \begin{align*}
        k^4 &> 4k^3 - 6k^2 + 4k - 1\\
        k^4 &> 12k^2 + 2
    \end{align*}
    The first inequality immediately follows from the fact that $(k - 1)^4 > 0$.
    The second inequality can be modified: $(k^2 - 6)^2 > 38$.
    It is obviously true.
    Add the two inequalities:
    \[
        2k^4 > 4k^3 + 6k^2 + 4k + 1.
    \]
    Then,
    \begin{align*}
        3^{k + 1} &= 3 \cdot 3^k\\
                  &> 3k^4 &\text{From the induction hyp.}\\
                  &= k^4 + 2k^4\\
                  &> k^4 + 4k^3 + 6k^2 + 4k + 1 &\text{As shown above}\\
                  &= (k + 1)^4
    \end{align*}
    Hence, by the induction principle, the given inequality holds for any $n \geq 9$.

    Therefore, the given inequality holds for $n = 1$ and $n \geq 9$.
\end{sol}

\begin{exercise}
    If $n \in \mathbb N$ and $x, y \geq 0$, then $\left(\frac{x + y}{2}\right)^n \leq \frac{x^n + y^n}{2}$.
\end{exercise}
\begin{proof}
    We use induction on $n$.
    When $n = 1$, $\left( \frac{x+y}{2}\right)^1 \leq \frac{x^1 + y^1}{2}$, so the given inequality holds.

    Suppose now as induction hypothesis that $\left( \frac{x + y}{2}\right)^k \leq \frac{x^k + y^k}{2}$ for some $k \geq 2$.
    Then,
    \begin{align*}
        \frac{x + y}{2} \left( \frac{x + y}{2}\right)^k &\leq \frac{x + y}{2} \cdot \frac{x^k + y^k}{2} &\text{From the induction hyp.}\\
                                                        &= \frac 14 \left[(x^{k+1} + y^{k+1}) + (x^k y + xy^k)\right]\\
                                                        &\leq \frac 14 \left[(x^{k+1} + y^{k+1}) + (x^{k+1} + y^{k+1})\right]. &\text{by Theorem~\ref{thm:rearrange}}\\
                                                        &= \frac{x^{k+1} + y^{k+1}}{2}.
    \end{align*}
    Hence, from the induction principle, the given inequality holds for all $n \in \mathbb N$.
\end{proof}

\begin{exercise}
    Two players move alternately in a game that starts with two equal-sized piles of coins.
    One move consists of removing some positive number of coins from one pile.
    The winner is the player who removes the last coin.
    Who will be the winner?
\end{exercise}
\begin{sol}
    The second player will win.
    The following is the winning strategy.

    \noindent\textbf{Case 1}: The first player removes all coins from one pile\\
    \indent Remove all coins from the other pile.

    \noindent\textbf{Case 2}: The first player removes some coins from one pile\\
    \indent Remove all coins except for one from the pile the first player removed some coins.
    Now there are two piles--one with one coin and one without a coin removed.
    Let the first pile be $A$, and the latter be $B$.
    The first player will now try to remove some coins from $B$, since taking a coin from $A$ would lead to the second player taking all coins from $B$, making the second player the winner.
    Now, the second player should remove all coins from $B$ except one, similarly as the last move.
    It is evident that the second player will win after this move.
\end{sol}

\begin{exercise}
    Let $\{a_n\}$ be a sequence satisfying $a_1 = 2, a_2 = 8$ and $a_n = 4(a_{n-1} - a_{n-2})$ for $n \geq 3$.
    Find a formula for $a_n$.
\end{exercise}
\begin{sol}
    Define a new sequence $b_n = a_n - 2a_{n-1}$ for $n \geq 2$.
    From the recurrence relation of $a_n$, it can be implied that $b_n = 2b_{n-1}$ for $n \geq 3$.
    Thus, $b_n = 2^{n-2} b_2 = 2^n$.
    Now we have
    \[
        a_n - 2a_{n-1} = 2^n.
    \]
    Then,
    \begin{align*}
        a_n - 2a_{n-1} &= 2^n,\\
        2(a_{n-1} - 2a_{n-2}) &= 2^n,\\
        \vdots\\
        2^{n-2} (a_2 - 2a_1) &= 2^n.
    \end{align*}
    Add all the equations:
    \[
        a_n - 2^{n-2} a_1 = 2^n (n - 1).
    \]
    Therefore, $a_n = 2^n n$.
\end{sol}

\begin{exercise}
    Show that the Fibonacci sequence $\{f_n\}$ have the following formula:
    \[
        f_n = \frac{\alpha^n - \beta^n}{\sqrt 5}
    \]
    where $\alpha = \frac{1 + \sqrt 5}{2}$ and $\beta = \frac{1 - \sqrt 5}{2}$.
\end{exercise}
\begin{proof}
    \begin{align*}
        f_1 &= \frac{1}{\sqrt 5} \left( \frac{1 + \sqrt 5}{2} - \frac{1 - \sqrt 5}{2}\right) = 1\\
        f_2 &= \frac{1}{\sqrt 5} \left( \left(\frac{1 + \sqrt 5}{2}\right)^2 - \left(\frac{1 - \sqrt 5}{2}\right)^2\right)\\
            &= \frac{1}{\sqrt 5} \left( \frac{6 + 2\sqrt 5}{4} - \frac{6 - 2 \sqrt 5}{4}\right) = 1\\
    \end{align*}
    Thus, we see that the initial conditions match that of the Fibonacci sequence.

    We will show that $f_{n+1} = f_{n} + f_{n-1}$ for $n \geq 2$.
    \begin{align*}
        f_{n + 1} &= \frac{\alpha^{n+1} - \beta^{n+1}}{\sqrt 5}\\
                  &= \frac{\frac{1 + \sqrt 5}{2} \alpha^n - \frac{1 - \sqrt 5}{2} \beta^n}{\sqrt 5}\\
                  &= \frac{\frac 12 (\alpha^n - \beta^n) + \frac{\sqrt 5}{2} (\alpha^n + \beta^n)}{\sqrt 5}\\
                  &= \frac{(\alpha^n - \beta^n) - \frac 12 (\alpha^n - \beta^n) + \frac{\sqrt 5}{2} (\alpha^n + \beta^n)}{\sqrt 5}\\
                  &= f_n + \frac{\frac{\sqrt 5 - 1}{2} \alpha^n + \frac{\sqrt 5 + 1}{2}\beta^n}{\sqrt 5}\\
                  &= f_n + \frac{\frac{\sqrt 5 - 1}{2} \cdot \frac{1 + \sqrt 5}{2} \alpha^{n-1} + \frac{\sqrt 5 + 1}{2} \cdot \frac{1 - \sqrt 5}{2} \beta^{n-1}}{\sqrt 5}\\
                  &= f_n + \frac{\alpha^{n-1} - \beta^{n-1}}{\sqrt 5}\\
                  &= f_n + f_{n-1}
    \end{align*}
    Therefore, the Fibonacci sequence follows the given formula.
\end{proof}
\end{document}
