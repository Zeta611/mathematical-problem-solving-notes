\documentclass[../main.tex]{subfiles}
\begin{document}
\subsection{Representation of Natural Numbers}
\begin{defn}
    Let $q$ be a natural number greater than 1.
    A \textsf{$q$-ary} or \textsf{base $q$ representation} of $n$ is a list $a_m, \dots, a_0$ of integers, each in $\{0, 1, \dots, q-1\}$, such that $a_m > 0$ and $n = \sum_{i = 0}^m a_i q^i$.
    For clarity, we may use a subscript $q$ to indicate that the base is $q$.
    We call representation in base 2, 3, and 10 as \textsf{binary}, \textsf{ternary}, and \textsf{decimal}, respectively.

    Theorem~\ref{thm:uniquerep} allows us to use the term \textbf{the} base $q$ representation, instead of \textbf{a} base $q$ representation.
\end{defn}

\begin{ex}
    The ternary representation for the first ten natural numbers in order are 1, 2, 10, 11, 12, 20, 21, 22, 100, 101.
    The corresponding representation in base 4 are 1, 2, 3, 10, 11, 12, 13, 20, 21, 22.
\end{ex}

\begin{thm}\label{thm:uniquerep}
    Let $q$ be a natural number greater than 1.
    Every natural number has a unique base $q$ representation with no leading zeros.
\end{thm}
\begin{proof}
    We use induction on $n$.
    For the case $n = 1$, $1 = 1 \cdot q^0$ is a base $q$ representation of $1$.

    Suppose now as an induction hypothesis that $n$ has a base $q$ representation $n = \sum_{i = 0}^m a_i q^i$.
    We consider two cases:\\
    \noindent \textbf{Case 1}: $a_0 = \dots = a_m = q - 1$\\
    \begin{align*}
        n &= \sum_{i = 0}^m a_i q^i\\
          &= \sum_{i = 0}^m (q - 1) q^i\\
          &= (q - 1) \sum_{i = 0}^m q^i\\
          &= (q - 1) \frac{q^{m+1} - 1}{q - 1}\\
          &= q^{m + 1} - 1
    \end{align*}
    Thus, $n + 1 = 1 \cdot q^{m + 1}$.

    \noindent \textbf{Case 2}: Otherwise\\
    \indent Let $t$ be the smallest index such that $a_t < q - 1$, i.e.,
    \[
        a_0 = \dots = a_{i - 1} = q - 1, a_t < q - 1, \dots
    \]
    Define $b_j$ as following:
    \[
        b_j = \begin{cases}
            a_j &\text{if } j > t\\
            a_t + 1 &\text{if } j = t\\
            0 &\text{if } j < t
        \end{cases}
    \]
    Then,
    \begin{align*}
        \sum_{i = 0}^m b_i q^i &= 0 + \dots + 0 + (a_t + 1) q^t + a_{t + 1} q^{t + 1} + \dots + a_m q^m\\
                               &= q^t + a_t q^t + a_{t+1} q^{t+1} + \dots + a_m q^m\\
                               &= q^t - 1 + 1 + a_t q^t + a_{t+1} q^{t+1} + \dots + a_m q^m\\
                               &= (q - 1) \sum_{i = 0}^{t - 1} q^i + 1 + a_t q^t + a_{t+1} q^{t+1} + \dots + a_m q^m\\
                               &= \sum_{i = 0}^{t - 1} (q - 1) q^i + 1 + a_t q^t + a_{t+1} q^{t+1} + \dots + a_m q^m\\
                               &= \sum_{i = 0}^{t - 1} a_i q^i + 1 + a_t q^t + a_{t+1} q^{t+1} + \dots + a_m q^m\\
                               &= \sum_{i = 0}^{m} a_i q^i + 1\\
                               &= n + 1
    \end{align*}
    Hence, from the induction principle, there is a $q$-ary representation for all $n \in \mathbb N$.

    Now we show the uniqueness of the base $q$ represenation.
    For the sake of contradiction, suppose there are two distinct $q$-ary representation of $n$.
    Let the two be $\sum_{i = 0}^s a_i q^i$ and $\sum_{i = 0}^t b_i q^i$.
    If $s \neq t$, let $s > t$ without loss of generality.
    Since $s, t \in \mathbb N$, $s \geq t+1$ can be implied.
    Then,
    \begin{align*}
        n &= \sum_{i=0}^s a_i q^i > q^s\\
        n &= \sum_{i=0}^t b_i q^i < q^{t+1} \leq q^s \
    \end{align*}
    We deduced $n > q^s \wedge n \leq q^s$. $\lightning$

    Therefore, $s = t$.
    Let $S = \{n | n \text{ has more than one distinct base $q$ representations.}\}$.
    Since $n$ has two distinct base $q$ representation, $n - q^s$ also has two distinct representations.
    Thus, for any element $e$ in $S$, where $q^m \leq e < q^{m+1}$, $e - q^m$ is also an element of $S$.
    As $S \subset \mathbb N$, this contradicts the well-ordering principle.
\end{proof}

From the following, we define $[n] = \{m | m \in \mathbb{N} \wedge m \leq n\} = \{1, 2, \dots, n\}$.

\begin{prob} [The Weights Problem]
    A balance scale has left and right pans; we can place objects in each pan and test whether the total weight is the same on each side.
    Suppose that five objects of known integer weight can be selected.
    How caan we choose the weights to guarantee being able to check all integer weights form 1 through 121?
    Given an object believed to have weight $n \in [121]$, how should we place the known weights to chekc it?
    Is it possible to choose five values to check more weights?
\end{prob}
\begin{sol}
    Intuitively, we can think of putting weights on the same side as the object as a negative weight.
    Thus, each weight can attribute 1, 0, -1, which corresponds to putting the weight on the other side of the pan, not putting the weight at all, and putting the weight on the same side, respectively.
    Since there are three possible attributes, we can come up with a ternary representation of the weight.
    We therefore choose $3^0=1, 3^1=3, 3^2=9, 3^3=27, 3^4=81$ as weights.

    Now show that our choice can actually weigh any object with weight $w \in [121]$, i.e., $(\forall w \in [121])(\exists (a_0, a_1, a_2, a_3, a_4) \in \{-1, 0, 1\}^5)\ w = \sum_{i = 0}^4 a_i 3^i$.
    From Theorem~\ref{thm:uniquerep}, there is a unique ternary representation of $u \in \{121, 122, \dots, 242\}$,
    \[
        u = \sum_{i = 0}^4 b_i 3^i
    \]
    where $b_i \in \{0, 1, 2\}$.
    Substracting $3^0 + 3^1 + \dots + 3^4 = 121$ to each side results in
    \[
        u - 121 = \sum_{i = 0}^4 (b_i - 1) 3^i.
    \]
    Note that $b_i - 1 \in \{-1, 0, 1\}$ and $u - 121 \in [121]$.
    Hence we can choose $a_i = b_i - 1$.
    Therefore, there always is a representation of $w$ as $\sum_{i=0}^4 a_i 3^i$.
\end{sol}

\subsection{Bijections}
\begin{defn}
    A function $f: A \rightarrow B$ is a \textsf{bijection} if for every $b \in B$ there is exactly one $x \in A$ such that $f(x) = b$.
    Such $f$ is called a \textsf{one-to-one correspondence} between $A$ and $B$.
\end{defn}

\begin{ex} \label{ex:NZbij}
    There is a bijection between $\mathbb N$ and $\mathbb Z$.
\end{ex}
\begin{proof}
    We give an explicit one-to-one correspondance from $\mathbb{N}$ to $\mathbb Z$.
    Correspond all even numbers to non-negative integers and odd numbers to negative integers:
    \[
        f(n) = \begin{cases}
        \frac n2 - 1 &\text{if $n$ is even.}\\
    - \frac{n + 1}{2} &\text{if $n$ is odd.}
\end{cases}
    \]
    We show that $f$ is a bijection.
    Note that $\frac n2 - 1 \geq 0$ for all $n \in \mathbb N$, and $- \frac{n+1}{2} <0$ for all $n \in \mathbb N$.
    Choose any integer $m \in \mathbb Z$.
    If $m \leq 0$, the only possible $n$ that corresponds to it must have a relation $-\frac{n+1}{2} = m$, and such $n$ exists uniquely as $-2m - 1$.
    If $m > 0$, the only possible $n$ that correponds to it must have a relation $\frac n2 - 1 = m$.
    We can determine $n$ uniquely as $2m + 2$.
    Therefore, $f$ is a bijection.
\end{proof}

\begin{ex}
    Given constants $a, b, c, d \in \mathbb{R}$, let $f: \mathbb{R}^2 \rightarrow \mathbb{R}^2$ is defined by $f(x, y) = (ax + by, cx + dy)$.
    When is $f$ a bijection?
\end{ex}
\begin{sol}
    For each $(\alpha, \beta) \in \mathbb{R}^2$, there needs to be $(x, y) \in \mathbb{R}^2$ such that $\alpha = ax + by$ and $\beta = cx + dy$.
    Using a matrix, we can rewrite the system as
    \[
        \begin{bmatrix}
            \alpha\\
            \beta
        \end{bmatrix}
        =
        \begin{bmatrix}
            a & b\\
            c & d
        \end{bmatrix}
        \begin{bmatrix}
            x\\
            y
        \end{bmatrix}.
    \]
    For $x$ and $y$ to exist uniquely, the inverse of
    \[
        \begin{bmatrix}
            a & b\\
            c & d
        \end{bmatrix}
    \]
    should exist.
    Therefore, the determinant being nonzero--$ad - bc \neq 0$--is both the necessary and sufficient condition for $f$ to be bijective.
\end{sol}

\begin{defn}
    If $f$ is a bijection from $A$ to $B$, then the \textsf{inverse} of $f$ is a function $g: B \rightarrow A$ such that for each $b \in B$, $g(b)$ is a unique element $x \in A$ such that $f(x) = b$.
    We write $f^{-1}$ for the function $g$.
\end{defn}

\begin{ex}
    If $f$ is a bijection and $g$ is the inverse of $f$, then $g$ is also a bijection and $f$ is the inverse of $g$.
    Thus $(f^{-1})^{-1} = f$.
\end{ex}
\begin{proof}
    Let $f$ be a function from $X$ to $Y$.
    Choose any $x_1 \in X$.
    Then there is a unique $y_1 \in Y$ such that $f(x_1) = y_1$, from the definition of a function (which was never formally stated).
    For such $y_1$, since $g$ is an inverse of $f$, i.e.,
    \[
        (\forall y \in Y)(\exists ! x \in X)\ g(y) = x,
    \]
    there is a unique $g(y_1) = x_2 \in X$ such that $f(x_2) = y_1 \in Y$.
    Since $f$ is a bijection, i.e.,
    \[
        (\forall y \in Y)(\exists ! x \in X)\ f(x) = y,
    \]
    and as $f(x_1) = f(x_2) = y_1$, it follows that $x_1 = x_2$.

    Therefore, $g$ is also a bijection as there is a unique $y \in Y$ for any $x \in X$.
\end{proof}

Showing that $f: A \rightarrow B$ is a bijection means showing that for each $b \in B$, the equation $f(x) = b$ has a unique solution in $A$.
Solving the equation to write a formula hat determines $x$ in terms of $b$ obtains a formula for $f^{-1}$.
We must check that the formula is valid on all of $B$.

\begin{ex}
    Define $f: \mathbb{R} \rightarrow \mathbb{R}$ by $f(x) = 5x - 2|x|$.
    Show that $f$ is a bijection.
\end{ex}
\begin{proof}
    Note that when $x \geq 0$, $f(x) = 3x \geq 0$, and when $x < 0$, $f(x) = 7x < 0$.
    Thus, $f(x)$ follows the sign of $x$ and vice versa.
    Choose any $y \geq 0$.
    If there is any, possible $x$ such that $f(x) = y$ is also greater or equal to 0.
    Hence we solve $3x = y$.
    $x$ is uniquely determined as $\frac y3$.

    Now choose any $y < 0$.
    Only possible $x$ such that $f(x) = y$ is negative.
    Thus solving $7x = y$, $x = \frac y7$.

    $x$ is uniquely determined for any $y \in \mathbb R$.
    Therefore, $f$ is a bijection.
\end{proof}

A bijection transforms elements of one set into elements of another, allowing us to work in either context.
For example, we can encode a subset $S$ of $[n]$ by recording the presence or absence of an element $i$ as 1 or 0 in position $i$ of an $n$-tuple $m(S)$.
An $n$-tuple with entries in $\{0, 1\}$ is a \textsf{binary $n$-tuple}; we call $m(s)$ the \textsf{binary encoding} of $S$.
From a binary $n$-tuple $b$, we will uniquely retrieve $S$ such that $m(S) = b$.
Thus binary encoding is a bijection from the power set of $[n]$ to the set of binary $n$-tuples.
\begin{ex}
    Consider the set $[3]$.
    For $S = \varnothing$, $m(S) = (0,0,0)$, and for $S = \{1, 3\}$, $m(S) = (1, 0, 1)$.
    Similarly, $S = \{1,2,3\}$ encodes to $m(S) = (1,1,1)$.
\end{ex}

\begin{prop}
    Binary encoding establishes a bijection from the power set of $[n]$ to the set of binary $n$-tuples.
\end{prop}
\begin{proof}
    We show that
    \[
        m: 2^{[n]} \rightarrow \mathbb{Z}_2^n
    \]
    is a bijection.

    Choose any $(b_1, \dots, b_n) \in \mathbb{Z}_2^n$.
    We can construct $S \subseteq [n]$ such that $i \in S$ iff $b_i = 1$.
    Thus, for any $(b_1, \dots, b_n) \in \mathbb{Z}_2^n$, we can find $S \in 2^{[n]}$.

    Now show that there is a unique subset of $[n]$ for a binary $n$-tuple.
    Suppose there are two distinct subsets $S$ and $T$ of $[n]$ for a binary encoding $b$ for the sake of contradiction.
    From a simple argument, we can show that
    \[
        \neg (p \Leftrightarrow q) \Leftrightarrow (p \wedge \neg q) \vee (q \wedge \neg p)
    \]
    Then, since $S \neq T$,
    \[
        (\exists i \in [n])\ (i \in S \wedge i \notin T) \vee (i \notin S \wedge i \in T).
    \]
    Without loss of generality, we only need to check the case $(\exists i \in [n])\ i \in S \wedge i \notin T$.
    Thus, $[m(S)]_i = 1$ whereas $[m(T)]_i = 0$.
    This contradicts the prior assumption that $b = m(S) = m(T)$. $\lightning$

    Therefore, $m$ is a bijection.
\end{proof}

\subsection{Injections and Surjections}
\begin{defn}
    A function $f: A \rightarrow B$ is \textsf{injection} if for each $b \in B$, there is at most one $x \in A$ such that $f(x) = b$.
    A function $f: A \rightarrow B$ is \textsf{surjective} if for each $b \in B$, there is at least one $x \in A$ such that $f(x) = b$.
    The corresponding nouns are \textsf{injection} and \textsf{surjection}.
\end{defn}

\begin{prop} \label{prop:strictinject}
    Let $f$ be a real-valued function defined on a subset of $\mathbb R$.
    If $f$ is strictly monotone, then $f$ is injective.
\end{prop}
\begin{proof}
    \[
        x_1 > x_2 \Rightarrow f(x_1) > f(x_2),
    \]
    since $f$ is strictly monotone.
    Then,
    \begin{align*}
        x_1 \neq x_2 &\Rightarrow x_1 > x_2 \wedge x_1 < x_2\\
                     &\Rightarrow f(x_1) > f(x_2) \wedge f(x_1) < f(x_2)\\
                     &\Rightarrow f(x_1) \neq f(x_2).
    \end{align*}
    Therefore, $f$ is injective.
\end{proof}

\begin{ex}
    What are the solutions to the equation below?
    \[
        x^4 + x^3 y + x^2 y^2 + x y^3 + y^4 = 0
    \]
\end{ex}
\begin{sol}
    Multiply each side of the given equation with $x - y$ yields:
    \[
        x^5 - y^5 = 0
    \]
    Since $f(x) = x^5$ is strictly monotone, from Proposition~\ref{prop:strictinject}, $f$ is an injection.
    Thus, if $x^5 = y^5$, $x = y$.
    Hence, the only possible solution for $(x - y)(x^4 + x^3 y + x^2 y^2 + x y^3 + y^4) = 0$ is $x = y$.
    Now substitute $x = y$ to the given equation:
    \[
        5x^4 = 0.
    \]
    Therefore, $x = y = 0$.
\end{sol}

\begin{prob}
    Show that if $n \in \mathbb N$ is an odd number, then $f(x) = x^n$ is strictly increasing.
\end{prob}
\begin{proof}
    Choose any odd number $n \in \mathbb N$.
    Then $(\exists m \in \mathbb{N} \cup \{0\})\ n = 2m + 1$.
    There are five cases:\\
    \noindent \textbf{Case 1}: $0 < x_1 < x_2$\\
    \indent If $0 < x_1 < x_2$, $0 < x_1^n < x_2^n$ follows since the two are in a positive set $(0, \infty) \subset \mathbb R$.

    \noindent \textbf{Case 2}: $x_1 < 0 < x_2$\\
    \indent $x_2^n$ is still positive since it is an element of a positive set.
    Since $-x_1 > 0$, $(-x_1)^{2m} = x_1^{2m} > 0$.
    Thus, $x_1^n = x_1^{2m} \cdot x_1 < 0$.
    Therefore, $x_1^n < x_2^n$.

    \noindent \textbf{Case 3}: $x_1 < x_2 < 0$\\
    \indent $0 < -x_2 < -x_1$, so from \textbf{Case 1}, $0 < (-x_2)^n < (-x_1)^n$.
    Since $(-x_1)^n = (-x_1)^{2m + 1} = (-x_1)^{2m} \cdot (-x_1) = -x_1^n$, and $(-x_2)^n$ is similarly $-x_2^n$, $0 < -x_2^n < -x_1^n$.
    Therefore, $x_1^n < x_2^n < 0$.

    Other two cases are when either $x_1$ or $x_2$ is 0, when $x_1 < x_2$.
    The two cases can be shown similarly.

    Therefore, $f(x) = x^n$ is strictly increasing.
\end{proof}

\subsection{Composition of Functions}
\begin{defn}
    If $f: A \rightarrow B$ and $g: B \rightarrow C$, then the \textsf{composition} of $g$ with $f$ is a function $h: A \rightarrow C$ defined by $h(x) = g\left(f(x)\right)$ for $x \in A$.
    When $h$ is a composition of $g$ with $f$, we write $h = g \circ f$.
\end{defn}

\begin{prop}
    \begin{enumerate}
        \item The composition of two injections is an injection.
        \item The composition of two surjections is a surjection.
        \item If $f$ and $g$ are bijections, then $(g \circ f)^{-1} = f^{-1} \circ g^{-1}$.
    \end{enumerate}
\end{prop}
\begin{proof}
    Let $f: A \rightarrow B$ and $g: B \rightarrow C$ be injections.

    1.
    \begin{align*}
        (g \circ f)(x_1) = (g \circ f)(x_2) &\Rightarrow g(f(x_1)) = g(f(x_2))\\
                                            &\Rightarrow f(x_1) = f(x_2)\\
                                            &\Rightarrow x_1 = x_2
    \end{align*}
    Thus, $g \circ f$ is also an injection.

    2. For each $c \in C$, there is $b \in B$ such that $g(b) = c$.
    Also, for each $b \in B$, there is $a \in A$ such that $f(a) = b$.
    Thus, For each $c \in C$, there is $a \in A$ such that $(g \circ f)(a) = g(f(a)) = g(b) = c$.
    Therefore, $g \circ f$ is also a surjection.

    3.
    \begin{align*}
        \left(f^{-1} \circ g^{-1}\right) \circ (g \circ f)(x) &= \left(f^{-1} \circ g^{-1}\right) \left((g \circ f)(x)\right)\\
                                                              &= \left(f^{-1} \circ g^{-1}\right) \left(g(f(x))\right)\\
                                                              &= f^{-1} \left(g^{-1}(g(f(x)))\right)\\
                                                              &= f^{-1} (f(x))\\
                                                              &= x
    \end{align*}
    Therefore, $f^{-1} \circ g^{-1} = (g \circ f)^{-1}$.
\end{proof}

\begin{prop} [Associativity of Composition]
    If $f: A \rightarrow B$, $g: B \rightarrow C$, and $h: C \rightarrow D$, then $h \circ (g \circ f) = (h \circ g) \circ f$.
\end{prop}
\begin{proof}
    \begin{align*}
        (h \circ (g \circ f))(x) &= h((g \circ f)(x))\\
                                 &= h(g(f(x)))\\
                                 &= (h \circ g)(f(x))\\
                                 &= ((h \circ g) \circ f(x))
    \end{align*}
    Therefore, $h \circ (g \circ f) = (h \circ g) \circ f$.
\end{proof}

\begin{ex}
    Let $S$ be the set of polynomials in one variable.
    Given the polynomial $f$ defined by $f(x) = \sum_{i = 0}^k a_i x^i$, let $Df$ denote the polynomial whose value $x$ is $\sum_{i=1}^k a_i i x^{i-1}$.
    The operator $D$ is a function $D: S \rightarrow S$.
    It is surjective; the polynomial with coefficients $\{a_k\}$ is the image of the polynomial $\sum_{i=0}^k \frac{a_i x^{i+1}}{i + 1}$.
    The operator $D$ is not injective; the polynomials $f$ and $g$ defined by $f(x) = x + 1$ and $g(x) = x + 2$ have the same image.

    We define another operator $J: S \rightarrow S$.
    For $f(x) = \sum_{i=0}^k a_i x^i$, let $Jf$ denote the polynomial whose value at $x$ is $\sum_{i=0}^k \frac{a_i x^{i+1}}{i + 1}$.
    If $Jf = Jg$, then term-by-term comparison of coefficients show that $f = g$; hence $J$ is injective.
    On the other hand, $J$ is not surjective, because there is no polynomial $f$ such that $Jf$ is a nonzero polynomial of degree 0.

    We can compose operators.
    We have $D(J(f)) = f$ for all $f \in S$, but $J(D(f))$ does not equal $f$ when $f(0) \neq 0$.
\end{ex}

\subsection{Cardinality}
\begin{defn}
    A set $A$ is \textsf{finite} if there is a bijection from $A$ to $[k]$ for some $k \in \mathbb{N} \cup \{0\}$.
    A set is \textsf{infinite} if there is no such bijection.
\end{defn}

\begin{prop} \label{prop:bij_card}
    If there is a bijection $f: [m] \rightarrow [n]$, then $m = n$.
\end{prop}
\begin{proof}
    We use induction on $n$.
    Consider the case $n = 1$.
    Since $f$ is a bijection, there should exist $x \in [m]$ such that $f(x) = 1$.
    For any $e \in [m]$, it should correspond with some element in $[1]$, which can only be $1$.
    Thus $f(e) = 1$.
    Since $f$ is injective, $f(e) = f(1) = 1$ implies $e = 1$ for any $e \in [m]$.
    Therefore, $m = 1$.

    Suppose now as an induction hypothesis that if there is a bijection $f: [m] \rightarrow [k]$, then $m = k$ for some $k \geq 2$.
    Let $g: [m] \rightarrow [k+1]$ be a bijection.\\
    \textbf{Case 1}: $g(m) = k+1$\\
    \indent Then $g|_{[m - 1]}: [m - 1] \rightarrow [k]$ is a bijection.
    Then from the induction hypothesis, $m - 1 = k$.
    Hence $m = k + 1$.

    \noindent \textbf{Case 2}: $g(m) \in [k]$\\
    \indent Let $g(m) = a \in [k]$.
    Now define $h: [k + 1] \rightarrow [k + 1]$ as the following:
    \[
        h(x) = \begin{cases}
            k + 1 &\text{if }x = a\\
            a &\text{if }x = k + 1\\
            x &\text{otherwise}.
        \end{cases}
    \]
    Then we see that $h \circ g$ is still a bijection, as $h$ and $g$ are both bijective.
    Note that $(h \circ g)(m) = h(g(m)) = h(a) = k + 1$.
    Since this was the case in \textbf{Case 1}, $m = k + 1$.

    Therefore, from the induction principle, existence of a bijection from $[m]$ to $[n]$ implies $m = n$.
\end{proof}

\begin{cor} \label{cor:finite_bij}
    If $A$ is finite, then for exactly one $n$, there is a bijection from $A$ to $[n]$.
\end{cor}
\begin{proof}
    Since $A$ is finite, there is a bijection $f: A \rightarrow [k]$ for some $k \in \mathbb{N} \cup \{0\}$.
    From Proposition~\ref{prop:bij_card}, $n = k$.
\end{proof}

\begin{defn}
    The \textsf{size} of a finite set $A$, written $|A|$, is the unique $n$ such that there is a bijection from $A$ to $[n]$.
    A set of size $n$ is called \textsf{$n$-element set} or \textsf{$n$-set}.
\end{defn}

\begin{cor} \label{cor:sum_disjoint}
    If $A$ and $B$ are disjoint finite sets, then
    \[
        |A \cup B| = |A| + |B|.
    \]
\end{cor}
\begin{proof}
    From Corollary~\ref{cor:finite_bij}, there are bijections $f: A \rightarrow [n]$, $g: B \rightarrow [m]$ for some $n, m \in \mathbb{N}$.
    Now construct $h: A \cup B \rightarrow [m + n]$ as following:
    \[
        h(x) = \begin{cases}
            f(x) &\text{if } x \in A\\
            g(x) + n &\text{if } x \in B\\
        \end{cases}.
    \]
    We need not consider the case $x \in A \wedge x \in B$ as $A$ and $B$ are disjoint.
    Note that $1 \leq f(x) \leq n$ and $n + 1 \leq g(x) + n \leq m + n$.
    We shall show that $h$ is a bijection.

    Choose any $k \in [m + n]$.
    If $k \leq m$, the only possible $x \in A \cup B$ to make $h(x) \leq n$ is the case when $x \in A$.
    Since $f$ is a bijection, there exists a unique $x \in A$ such that $f(x) = h(x) = k$.

    Similarly, if $m < k \leq m + n$, the only possible case is when $x \in B$.
    There exists $l \in [n]$ such that $m + l = n$.
    Also, there exists a unique $x \in B$ such that $g(x) = l$ since $g$ is a bijection.
    Thus, such $x$ satisfies the condition $h(x) = l + m = k$.

    Therefore, there is a unique $x \in A \cup B$ such that $h(x) = k$ implying $h$ is a bijection.
\end{proof}

\begin{cor}
    Every nonempty finite set of real numbers has both a maximum element and a minimum element.
\end{cor}
\begin{proof}
    We use induction on the size of a set $S$, $|S| = n$.
    For the case $n = 1$, the only element of $S$ should both be the maximum and the minimum.

    Now as an induction hypothesis, suppose that a set of $k$ elements has both a maximum element and a minimum element, for some $k \geq 2$.
    Cosider a set $S$ where $|S| = k + 1$.
    Choose any element $e \in S$.
    Then from the induction hypothesis, there is both a maximum element $M$ and a minimum element $m$ of $S \backslash \{e\}$.
    We see that $\max (\{e, M\})$ and $\min (\{e, m\})$ are the maximum and the minimum element of $S$, respectively.
    Thus, there is both a maximum and a minimum element of any nonempty finite set of real numbers.
\end{proof}

\begin{defn}
    An infinite set $A$ is \textsf{countably infinite} (or \textsf{countable}) if there is a bijection from $A$ to $\mathbb N$; otherwise $A$ is \textsf{uncountably infinite} (or \textsf{uncountable}).
    Sets $A$ and $B$ have the same \textsf{cardinality} if there is a bijection from $A$ to $B$.
\end{defn}

\begin{ex}
    $\mathbb Z$ and $\mathbb Q$ are countable, whereas $\mathbb R$ is not.
\end{ex}
\begin{proof}
    From Example~\ref{ex:NZbij}, $\mathbb Z$ is countable.
    Now consider the following sequence:
    \[
        \frac 01, \frac 11, -\frac 11, \frac 12, \frac 21, -\frac 12, -\frac 21, \frac 13, \frac 23, \frac 31, \frac 32, -\frac 13, -\frac 23, -\frac 31, -\frac 32, \frac 14, \dots
    \]
    It is evident that all rational numbers will appear at the above sequence.
    Since we can see a sequence as a bijective function from $\mathbb N$ to the elements of the sequence, the above sequence shows the bijection from $\mathbb N$ to $\mathbb Q$.
    Thus $\mathbb Q$ is countable.
    
    We now show the uncountability of $\mathbb R$ using the famous diagonal argument from Cantor.
    Every real number in $(0, 1]$ can be represented in the form $0.a_1 a_2 a_3 \dots$ uniquely.
    Note that real number like $0.1$ should be written in form $0.999\cdots$ to ensure the uniqueness of such representation.
    For the sake of contradiction, suppose there is a bijection $f$ from $\mathbb{N}$ to $\mathbb{R}$:
    \begin{align*}
        f(1) &= 0.a_{11} a_{12} a_{13} \dots\\
        f(2) &= 0.a_{21} a_{22} a_{23} \dots\\
        \vdots \\
        f(k) &= 0.a_{k1} a_{k2} a_{k3} \dots\\
        \vdots
    \end{align*}
    Now consider the number $b \in (0, 1]$ defined as the following:
    \[
        b = 0.b_1 b_2 b_3 \dots \text{, where } b_i = \begin{cases}
            1 &\text{if } a_{ii} \neq 1\\
            0 &\text{if } a_{ii} = 1
        \end{cases}
    \]
    Such $b$ is never equal with $f(k)$ for all $k \in \mathbb N$, as $b_i \neq a_{ii}$ for all $i \in \mathbb N$.
    However, $f$ is a bijection. $\lightning$
    
    Therefore, no bijection from $\mathbb N$ to $\mathbb R$ exists, so $\mathbb R$ is uncountably infinite.
\end{proof}

\begin{thm}
    The sets $\mathbb{N} \times \mathbb{N}$ and $\mathbb N$ has the same cardinality.
    That is, $\mathbb{N} \times \mathbb{N}$ is countable.
\end{thm}
\begin{proof}
    We can construct the following sequence:
    \[
        (1,1), (2,1), (1,2), (3,1), (2,2), (1,3), (4,1), (3,2), \dots
    \]
    It is evident that all elements in $\mathbb{N} \times \mathbb{N}$ are included in the above sequence.
    Since it is a sequence, there is a bijection from $\mathbb{N}$ to $\mathbb{N} \times \mathbb{N}$.
\end{proof}

\begin{prob}
    Show that $(0, 1)$ and $\mathbb R$ have the same cardinality.
\end{prob}
\begin{proof}
    We can construct a bijection from $(0, 1)$ to each point of $x^2 + (y-1)^2 = 1$ where $y < 1$.
    Simply consider the angle created by the point $(x, y)$ on the half-circumference, $(0, 1)$, and $(1,1)$, which in $(0, \pi)$.
    Thus, consider the bijection $f(x) = \pi x$, which is a bijection between the angle and $(0, 1)$.
    Then the angle makes a bijection to the half-circumference.
    Now consider the intersection of the half-line connecting $(x, y)$ from $(0,1)$ and the $x$-axis.
    We see that it creates a bijection from each point of the half-circumference and the point on the $x$-axis.
    Thus, there is a bijection from the half-circumference to $\mathbb R$.
    Therefore, there is a bijection from $(0, 1)$ to $\mathbb R$, implying that the cardinalities of the two sets are equal.
\end{proof}

\begin{lem} \label{lem:forSchroeder}
    For two sets $A$ and $B$, if $B \subseteq A$ and $f: A \rightarrow B$ is injective, there exists a bijection between $A$ and $B$.
\end{lem}
\begin{proof}
    Let $Y = A - B$ and $X = Y \cup f(Y) \cup f^2(Y) \cup \dots$.
    Then $Y \cap B = \varnothing$ and $f^k (Y) \subseteq B$.
    Thus, 
    \[
        (\forall k \geq 1)\ Y \cap f^k (Y) = \varnothing.
    \]
    Since $f$ is an injection, $f^m$ is also an injection for any $m \geq 1$.
    Then we have
    \[
        (\forall k, m \geq 1)\ f^m(Y) \cap f^m \left(f^k(Y)\right) = f^m(Y) \cap f^{m + k} (Y) = \varnothing.
    \]
    Thus, all $f^k(Y)$'s are disjoint for all $k \geq 0$.

    From $X = Y \cup f(Y) \cup f^2(Y) \cup \dots$, we have $f(X) = f(Y) \cup f^2(Y) \cup f^3(Y) \cup \dots \subseteq B$.
    Since $f$ is injective, $f|_X: X \rightarrow f(X)$ is a bijection.
    Also, we have
    \[
        A - X = (B \cup Y) - (Y \cup f(X)) = B - f(X).
    \]
    Define $h: A \rightarrow B$ by:
    \[
        h(z) = \begin{cases}
            f(z) &\text{if } z \in X\\
            z &\text{if } z \in A - X
        \end{cases}
    \]
    By definition, $h$ is injective.
    Let $b \in B$ be given.
    If $b \in f(X)$, then $f(z) = b$ for some $z \in X$.
    Then $h(z) = f(z) = b$.
    If $b \in B - f(X)$, then $b \in A - X$, implying $h(b) = b$.
    Hence $h$ is a surjection.
    Therefore, $h$ is a bijection from $A$ to $B$
\end{proof}

\begin{thm} [Schr\"oder-Bernstein Theorem] \label{thm:shroder}
    If $f: A \rightarrow B$ and $g: B \rightarrow A$ are injections, then there exists a bijection $h: A \rightarrow B$, and hence $A$ and $B$ have the same cardinality.
\end{thm}
\begin{proof}
    $g: B \rightarrow A$, so $g(B) \subseteq A$.
    Now consider $g \circ f: A \rightarrow g(B)$.
    Since both $f$ and $g$ are injections, $g \circ f$ is also an injection.
    From Lemma~\ref{lem:forSchroeder}, there is a bijection between $A$ and $g(B)$.
    Let such bijection be $h'$.
    Now define $h = g^{-1} \circ h': A \rightarrow B$.
    Note that $g^{-1}: g(B) \rightarrow B$ is a bijection as $g$ is an injection.
    Thus, $h$, a composition of two bijections is also a bijection.
\end{proof}

\subsection{Exercises for Chapter 5}
\begin{exercise}
    Let $f: A \rightarrow B$ be a function.
    Prove that there is a function $g: B \rightarrow A$ such that $f \circ g = I_B$ iff $f$ is a surjection.
\end{exercise}
\begin{proof}
    We need to show that for functions $f: A \rightarrow B$,
    \[
        (\exists g: B \rightarrow A) (\forall x \in B)\ (f \circ g)(x) = x \Leftrightarrow (\forall y \in B)(\exists x \in A)\ f(x) = y.
    \]

    We first show the $\Rightarrow$ direction of the statement.
    Choose any $y \in B$. 
    Since $(\forall x \in B)\ (f \circ g)(x) = x$, $(f \circ g)(y) = f(g(y)) = y$.
    Thus we have $x = g(y) \in A$ such that $f(x) = y$.
    Therefore, 
    \[
        (\exists g: B \rightarrow A) (\forall x \in B)\ (f \circ g)(x) = x \Rightarrow (\forall y \in B)(\exists x \in A)\ f(x) = y.
    \]

    Now we show the $\Leftarrow$ direction of the statement.
    Since there is always a corresponding $x \in A$ to any $y \in B$ since $f$ is surjective, we can define a nonempty set $S(y) \subset A$ such that $S(y) = \{e \in A | f(e) = y\}$.
    We construct a function $g: B \rightarrow A$:
    \[
        g(x) = a,
    \]
    where $a$ is an arbitrary element from $S(x) \neq \varnothing$.
    Then,
    \begin{align*}
        (f \circ g)(x) &= f(g(x))\\
                       &= f(a) &\text{where $a \in S(x) = \{e \in A | f(e) = x\}$}\\
                       &= x.
    \end{align*}
    Thus, $f\circ g = I_B$.
    Therefore,
    \[
        (\exists g: B \rightarrow A) (\forall x \in B)\ (f \circ g)(x) = x \Leftarrow (\forall y \in B)(\exists x \in A)\ f(x) = y.
    \]

    We showed both direction of the given statement.
\end{proof}

\begin{exercise}
    Suppose that $f: X \rightarrow Y$ is a function.
    Define $\overrightarrow f: \mathcal{P}(X) \rightarrow \mathcal{P}(Y)$ by
    \[
        \overrightarrow f(A) = \{f(x) | x \in A\},
    \]
    for $A \in \mathcal{P}(X)$.
    Similarly, define $\overleftarrow f: \mathcal{P}(Y) \rightarrow \mathcal{P}(X)$ by
    \[
        \overleftarrow f(B) = \{x \in X | f(x) \in B\},
    \]
    for $B \in \mathcal{P} (Y)$.

    Prove that the followings are equivalent:
    \begin{enumerate}
        \item $f$ is injective.
        \item $\overrightarrow f$ is injective.
        \item $\overleftarrow f$ is surjective.
    \end{enumerate}
\end{exercise}
\begin{proof}
    We first show that if $f$ is injective, then $\overrightarrow f$ is injective.
    Choose any two distinct $X_1$ and $X_2$ from $\mathcal{P}(X)$.
    If one of $X_1$ and $X_2$ is an empty set, let it be $X_1$, without loss of generality.
    Then $\overrightarrow{f}(X_1) = \overrightarrow{f}(\varnothing) = \varnothing$, and $\overrightarrow{f}(X_2) \neq \varnothing$, so $f(X_1) \neq f(X_2)$.
    Now consider the case when both $X_1$ and $X_2$ are not empty.
    Note that $X_1 \neq X_2$ implies the following:
    \[
        X_1 - X_2 \neq \varnothing \vee X_2 - X_1 \neq \varnothing.
    \]
    Without loss of generality, assume $X_1 - X_2 \neq \varnothing$.
    Then, there is $x_1 \in X_1 - X_2$.
    There is no other element in $X$ other than $x_1$ such that $f(x) = f(x_1)$, since $f$ is injective.
    Thus $f(x_1) \in \overrightarrow{f}(X_1) - \overrightarrow{f}(X_2)$.
    Therefore, $\overrightarrow{f}(X_1) \neq \overrightarrow{f}(X_2)$, implying $\overrightarrow{f}$ is injective.

    Now show that if $\overrightarrow f$ is injective, then $f$ is injective as well.
    Choose any two distinct $x_1$ and $x_2$ from $X$.
    Consider $\{x_1\}$ and $\{x_2\}$ from $\mathcal{P}(X)$.
    Since $\overrightarrow f$ is injective,
    \begin{align*}
        \overrightarrow{f}(\{x_1\}) \neq \overrightarrow{f}(\{x_2\}) &\Rightarrow \{f(x_1)\} \neq \{f(x_2)\}\\
                                                                     &\Rightarrow f(x_1) \neq f(x_2)
    \end{align*}
    Therefore, $f$ is injective.

    We shall show that if $f$ is injective, then $\overleftarrow{f}$ is surjective.
    Choose any $Y_1$ from $\mathcal{P}(Y)$.
    If $Y_1 = \varnothing$, then $\overleftarrow{f}(Y_1) = \overleftarrow{f}(\varnothing) = \varnothing \in \mathcal{P}(X)$.
    Now consider the case when $Y_1$ is nonempty.
    For any $y \in Y_1$, there are two cases: there is a unique $x \in X$ such that $f(x) = y$, or there is no $x \in X$ at all such that $f(x) = y$.
    For the either case, $\{x | f(x) \in Y\} \in \mathcal{P}(X)$.
    Thus, $\overleftarrow{f}$ is surjective.

    Finally, show that $\overleftarrow{f}$ being a surjection implies that $f$ is injective.
    For the sake of contradiction, suppose $f$ is not injective.
    Then for some $y_1 \in Y$, there are two distinct $x_1$ and $x_2$ in $X$ such that $f(x_1) = f(x_2) = y_1$.
    Since $\overleftarrow{f}$ is surjective, there is $Y_1, Y_2, Y_3 \in \mathcal{P}(Y)$ such that $\overleftarrow{f}(Y_1) = \{x_1\}$ and $\overleftarrow{f}(Y_2) = \{x_2\}$.
    Only $y \in Y$ such that $y = f(x_1) = f(x_2)$ is $y = y_1$ from the definition of a function.
    Thus, $y_1 \in Y_1$ and $y_1 \in Y_2$.
    However, if $y_1 \in Y_1$, then $x_2 \in \overleftarrow{f}(Y_1)$ from definition of $\overleftarrow{f}$. $\lightning$
    Therefore, $f$ is injective if $\overleftarrow{f}$ is a surjection.

    We now have that all three expressions are equivalent.
\end{proof}

\begin{exercise}
    Let $B$ be a proper subset of a set $A$, and let $f$ be a bijection from $A$ to $B$.
    Prove that $A$ is an infinite set.
\end{exercise}
\begin{proof}
    For the sake of contradiction, suppose that $A$ is finite, i.e., there is a bijection $h: A \rightarrow [n]$ for some $n \in \mathbb{N}$.
    Note that the existance of a proper subset of $A$ ensures that $A$ is nonempty.
    
    $B$ is a proper subset of $A$, so $\exists a \in A - B \neq \varnothing$.
    Then we see that there is no bijection from $B$ to $[n]$.
    However, we do have a bijection $f$ from $A$ to $B$ and a bijection $h$ from $A$ to $[n]$. $\lightning$

    Hence the assumption that $A$ is finite is wrong.
    Therefore, $A$ is an infinite set.
\end{proof}

\begin{exercise}
    Let $f$ be a function from $A$ to itself.
    Prove that $f$ is injective iff $f$ is surjective.
    Prove that this equivalence fails when $A$ is infinite.
\end{exercise}
\begin{proof}
    Consider the case when $A$ is finite.

    We first show that if $f$ is injective, then $f$ is surjective.
    Define a function $g: A \rightarrow f(A)$, where $g(x) = f(x)$.
    Then $g$ is a bijection, since $f$ is injective and $f(A)$ is now the codomain of $g$.
    Thus, $|A| = |f(A)|$.
    We also have $f(A) \subseteq A$, so $A = f(A)$.
    Therefore $f = g$, implying $f$ is also a bijection, thus surjective as well.

    Now show that if $f$ is surjective, then $f$ is injective.
    For the sake of contradiction, suppose 
    \[
        (\exists y_1 \in A)(\exists x_1, x_2 \in A)\ f(x_1) = f(x_2) = y_1.
    \]
    Then for a set $S = \{x | f(x) = y_1\}$, $|S| \geq 2$.
    Define a function $h: A \backslash S \rightarrow A \backslash \{y_1\}$, where $h(x) = f(x)$.
    Choose any $y^* \in A \backslash \{y_1\}$.
    Since $f$ is surjective, there is $x^* \in A$ such that $f(x^*) = y^*$.
    $A \backslash S = \{ x | f(x) \neq y_1\}$, so $x^* \in A \backslash S$.
    Thus, 
    \[
        (\forall y \in A \backslash \{y_1\})(\exists x \in A \backslash S)\ f(x) = h(x) = y
    \]
    Hence $h$ is surjective.
    However, $|A \backslash \{y_1\}| = |A| - 1$ whereas $|A \backslash S| \leq |A| - 2$.
    With a cardinality of the domain smaller than the codomain, constructing a surjection is impossible. $\lightning$

    Hence our previous assumption that $f$ is not an injection is wrong.
    Therefore, if $f$ is surjective, then $f$ is injective.

    We have shown that when $A$ is a finite set, $f$ is injective iff $f$ is surjective.
    However, for the case when $A$ is an infinite set, the statement does not hold.
    For instance, consider the function $f: \mathbb{N} \rightarrow \mathbb{N}$, defined by $f(n) = 2n$.
    $f$ is clearly an injection, yet it is not a surjection.
\end{proof}

\begin{exercise}
    Let $A_1, A_2, \dots$ be a sequence of sets, each of which is countable.
    Prove that the union of all the sets in the sequence is a countable set.
\end{exercise}
\begin{proof}
    Before we prove the statement, we show that a union of two countable sets is countable as well.

    Assume two sets $A$ and $B$ are countable.
    Then there exist bijections $f_A: A \rightarrow \mathbb N$ and $f_B: B \rightarrow \mathbb N$.
    Now define a function $f: A \cup B \rightarrow \mathbb N$:
    \[
        f(x) = \begin{cases}
            2f_A{x} &\text{if } x \in A\\
            2f_B(x) + 1 &\text{if } x \in B \backslash A
        \end{cases}
    \]
    $f$ is injective, as $x \in A$ corresponds to even numbers and $f_A$ is injective, whereas $x \in B \backslash A$ corresponds to odd numbers and $f_B$ is injective.

    On the other hand, $f_A: A \rightarrow \mathbb N$ is a bijection.
    Hence, $f_A^{-1}: \mathbb{N} \rightarrow A$ is an injection.
    Let $g: \mathbb{N} \rightarrow A \cup B$ be defined as $g(x) = f_A^{-1}(x)$.
    Then $g$ is an injection.

    We have shown that there are injections both from $A \cup B$ to $\mathbb N$ and $\mathbb N$ to $A \cup B$.
    From Theorem~\ref{thm:shroder}, $A \cup B$ and $\mathbb N$ have the same cardinality, i.e., $A \cup B$ is countable.

    Now we prove the given statement, and we use induction on the length of the sequence $n$.
    For the case $n = 1$, when $A_1$ is countable, union of all sets in the sequence, which is $A_1$ itself, is countable.
    Suppose as an induction hypothesis that for some $k \geq 2$, if $A_i$ is countable for all $i \in [k]$, then $\bigcup_{i=1}^n A_i$ is countable.
    Consider $\bigcup_{i=1}^{n+1} A_i$ when all $A_i$'s are countable for $i \in [k+1]$.
    From the induction hypothesis, $\bigcup_{i=1}^n A_i$ is countable.
    We have shown above that a union of two countable sets is countable.
    $\bigcup_{i=1}^{n+1} A_i = \bigcup_{i=1}^n \cup A_{n+1}$ is thus countable.
    Therefore, from the induction principle, a union of all the sets in a sequence of coutable sets is countable.
\end{proof}

\begin{exercise}
    Construct an explicit bijection from the open interval $(0, 1)$ to the closed interval $[0, 1]$.
\end{exercise}
\begin{proof}
    Construct a sequence $\left< a_i \right>$ of rational numbers in $(0, 1)$:
    \[
        a_1 = \frac 12, a_2 = \frac 13, a_3 = \frac 23, a_4 = \frac 14, a_5 = \frac 34, a_6 = \frac 15, a_7 = \frac 25, a_8 = \frac 35, a_9 = \frac 45, a_{10} = \frac 16, \dots
    \]
    Any rational number in $(0, 1)$ is included in the above sequence only once.
    Now define a function $f: [0, 1] \rightarrow (0, 1)$:
    \[
        f(x) = \begin{cases}
            a_1 &\text{if } x = 0\\
            a_2 &\text{if } x = 1\\
            a_{i + 2} &\text{if } x = a_i\\
            x &\text{if } x \notin \mathbb{Q}
        \end{cases}
    \]
    For any $y \in (0, 1)$, $y$ is either in $(0, 1) \cap \mathbb Q$ or $(0, 1) - \mathbb Q$.
    Choose any $y^* \in (0, 1) \cap \mathbb Q$.
    Then $(\exists ! k \in \mathbb{N})\ a_k = y^*$.
    When $k = 1$, the only corresponding $x$ such that $f(x) = y^*$ is $x = 0$.
    When $k = 2$, the only corresponding $x$ such that $f(x) = y^*$ is $x = 1$.
    When $k \geq 3$, corresponding $x$ such that $f(x) = y^*$ is uniquely determined as $x = a_{k - 2}$.
    Now when $y^* \in (0, 1) - \mathbb{Q}$, only $x$ such that $f(x) = y^*$ is $x = y^*$.
    Therefore, proposed $f$ is bijective.
\end{proof}
\end{document}
