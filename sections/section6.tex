\documentclass[../main.tex]{subfiles}
\begin{document}
\subsection{Arrangements and Selection}
\begin{defn}
    A \textsf{partition} of a set $A$ is a collection of pairwise disjoint subsets of $A$ whose union is $A$.
    The \textsf{rule of sum} states that if $A$ is finite and $B_1, \dots, B_m$ is a partition of $A$, then $|A| = \sum_{i = 1}^m |B_i|$.
\end{defn}

\begin{prob}
    Prove the rule of sum.
\end{prob}
\begin{proof}
    The rule of sum follows directly from Corollary~\ref{cor:sum_disjoint}.
\end{proof}

\begin{defn}
    Let $T$ be a set whose elements can be described using a procedure involving steps $S_1, \dots, S_k$ such that step $S_i$ can be performed in $r_i$ ways, regardless of how steps $S_1, \dots, S_{i-1}$ are performed.
    The \textsf{rule of product} states that $|T| = \prod_{i=1}^k r_i$.
\end{defn}

\begin{prob}
    Prove the rule of product.
\end{prob}
\begin{proof}
    We use induction $n$, the number of procedures needed to describe elements in $T$.
    For the case $n = 1$, $|T| = r_1$, so the given statement holds.
    Assume now as an induction hypothesis that $|T| = \prod_{i = 1}^k r_i$ for some $k \geq 2$.
    Now when $k + 1$ steps are needed to perform elements of $T$, $k$ steps should be performed before the $(k+1)$-th one.
    Performing $k + 1$ steps before the $(k+1)$-th one takes $R = \prod_{i=1}^k r_i$ ways from the induction hypothesis.
    Since each way of performing the $(k+1)$-th step is distinct, the rule of sum applies:
    \[
        |T| = \sum_{i=1}^{R} r_{k+1} = Rr_{k+1} = \prod_{i=1}^{k+1} r_i.
    \]
    Therefore, the rule of product holds for any number of steps from the induction principle.
\end{proof}

\begin{defn}
    A \textsf{permutation} of a finite set $S$ is a bijection from $S$ to itself.
    The \textsf{word form} of a permutation of $[n]$ is the list obtained by writing the image of $i$ in position $i$.
\end{defn}
\begin{ex}
    The word form of a permutation simply records the function; for example, $f: [3] \rightarrow [3]$ defined by $f(1) = 2$, $f(2) = 3$, and $f(3) = 1$ is the permutation with word form 231.
    We often use the term permutation for botht the function and the word form.
\end{ex}

In counting problems, we use the word \textsf{arrangements} to refer to lists formed from a specified set.
We generalize permutations by considering arrangements without repetition.

\begin{thm} \label{thm:perm}
    An $n$-element set has $n!$ permutations (arrangements without repetition).
    In general, the number of arrangements of $k$ distinct elements from a set of size $n$ is $n(n -1) \dots (n - k + 1)$.
\end{thm}
\begin{proof}
    This is a direct result of the rule of product.
\end{proof}

\begin{defn}
    A \textsf{selection} of $k$ elements from $[n]$ is a $k$-element subset of $[n]$.
    The number of such selection is $n$ \textsf{choose} $k$, written as $n \choose k$.
    If $k < 0$ or $k > n$, $n \choose k$ is 0; in these cases, there are no selection of $k$ elements from $[n]$.
\end{defn}

\begin{thm} \label{thm:comb}
    For integer $n$, $k$ with $0 \leq k \leq n$,
    \[
        {n \choose k} = \frac{n!}{k! (n - k)!}.
    \]
\end{thm}
\begin{proof}
    From Theorem~\ref{thm:perm}, there are $n(n-1) \dots (n-k+1)$ permutations of length $k$ from $n$ elements.
    This is equal to selecting $k$ elements from $n$ elements and making a permutation out of the selection, i.e., ${n \choose k} \cdot k!$.
    Thus, we have the following relation:
    \[
        n(n-1) \dots (n-k+1) = \frac{n!}{(n-k)!} = {n \choose k} \cdot k!.
    \]
    Therefore, ${n \choose k} = \frac{n!}{k! (n-k)!}$.
\end{proof}

\begin{prob} [Comparison of Poker Hands]
    A poker hand consists of five cards from an ordinary deck of cards.
    Find the number of cases for the following:
    \begin{enumerate}
        \item One pair
        \item Two pair
        \item Three of a kind
        \item Straight
        \item Flush
        \item Full house
        \item Four of a kind
        \item Straight flush
    \end{enumerate}
\end{prob}
\begin{sol}
    \noindent \textbf{1. One pair}\\
    \indent Choose a rank of the pair: 13.
    Choose two suits of the pair: $4 \choose 2$.
    Choose ranks of the rest of the cards: $12 \choose 3$.
    Choose suits of the rest of the cards: $4^3$.
    The overall number of cases is ${13 \choose 1}{4 \choose 2}{12 \choose 3}4^3 = 1,098,240$.

    \noindent \textbf{2. Two pair}\\
    \indent Choose ranks of the two pairs: $13 \choose 2$.
    Choose two suits of the pairs: ${4 \choose 2}^2$.
    Choose a rank of the left card: 11.
    Choose a suit of the left card: 4.
    The overall number of cases is ${13 \choose 2}{4 \choose 2}^2 \cdot 11 \cdot 4 = 123,552$

    \noindent \textbf{3. Three of a kind}\\
    \indent Choose a rank of the three: 13.
    Choose three suits of the three: $4 \choose 3$.
    Choose ranks of the other two cards: $12 \choose 2$.
    Choose suits of the other two cards: $4^2$.
    The overall number of cases is $13 \cdot {4 \choose 3} {12 \choose 2} \cdot 4^2 = 54,912$.

    \noindent \textbf{4. Straight}\\
    \indent Choose the consecutive ranks (ace to 5, 10 to ace): 10.
    Choose a suit for each card, but all are not the same: $4^5 - 4$.
    The overall number of cases is: $10(4^5 - 4) = 10,200$.

    \noindent \textbf{5. Flush}\\
    \indent Choose the suit for the flush: 4.
    Choose a rank for each card, but not consecutive: ${13 \choose 5} - 10$.
    The overall number of cases is $4 \left({13 \choose 5} - 10\right) = 5,108$.

    \noindent \textbf{6. Full house}\\
    \indent Choose ranks for the three and the pair: $13 \cdot 12$.
    Choose a suit for the three: $4 \choose 3$.
    Choose a suit for the pair: $4 \choose 2$.
    The overall number of cases is $13 \cdot 12 \cdot {4 \choose 3}{4 \choose 2} = 3,744$.

    \noindent \textbf{7. Four of a kind}\\
    \indent Choose a rank for the four: 13.
    Choose the left card: 48.
    The overall number of cases is: $13 \cdot 48 = 624$.

    \noindent \textbf{8. Straight flush}\\
    \indent Choose the consecutive ranks (ace to 5, 10 to ace): 10.
    Choose a suit for the flush: 4.
    The overall number of cases is $10 \cdot 4 = 40$.
\end{sol}

The number $n \choose k$ is called the \textbf{binomial coefficient} due to its appearance as a coefficient in the $n$-th power of a sum of two terms.
\begin{thm} [Binomial Theorem]
    \[
        (x + y)^n = \sum_{i = 0}^n {n \choose i} x^i y^{n-i}
    \]
\end{thm}
\begin{proof}
    We can choose either $x$ or $y$ from $n$ terms of $x + y$.
    The number of cases to choose $x$, $i$ times and $y$, $(n-1)$ times is ${n \choose i}{n-i \choose n-i} = {n \choose i}$.
    Therefore,
    \[
        (x + y)^n = \sum_{i=0}^n {n \choose i} x^i y^{n-i}.
    \]
\end{proof}

\subsection{Binomial Coefficients}
\begin{lem}
    \[
        {n \choose k} = {n \choose n - k}
    \]
\end{lem}
\begin{proof}
    Consider a set $S$ where $|S| = n$.
    Choosing a subset $T$ with $k$ elements is equal to choosing a subset $S - T$ with $n - k$ elements.
    Therefore, ${n \choose k} = {n \choose n - k}$.
\end{proof}

\begin{defn}
    A \textsf{lattice path} in the plane is a path joining integer points via steps of unit length rightward or upward.
    Alternatively, it is a list of ordered pairs of integers, with each step increasing one coordinate by 1.
    The \textsf{length} of a path is the total number of steps.
\end{defn}

\begin{ex} [Lattice Paths and Binary Lists]
    We start lattice paths at the origin.
    Since each step increases a coordinate by 1, the length of the walk is the sum of the coordinates of the ending point.
    We can encode a path by recording in position $i$ as 1, when the $i$-th step is rightward and 0, when the $i$-th step is upward.
    In a path of length $n$, if there are $k$ steps to the right, we reach the point $(k, n - k)$, and the encoding has $k$ 1's.

    Furthermore, the actual path is determined by which steps are taken to the right.
    Thus the path is determined by the binary $n$-tuple.
    This establishes a one-to-one correspondence between the lattice paths reaching $(k, n - k)$ and the binary $n$-tuples with $k$ 1's.
    Hence the number of a lattice path $(k, n - k)$ is $n \choose k$.
\end{ex}

\begin{prop}
    For nonnegative integers $a$ and $b$, the number of lattice paths from the origin to the point $(a, b)$ is $a + b \choose a$.
\end{prop}
\begin{proof}
    Lattice paths from the origin to the point $(a, b)$ is composed of $a$ 1's and $b$ 0's.
    Since there is a one-to-one correspondence between each path and binary $(a + b)$-tuples with $a$ 1's, we can simply count the number of the binary tuples.
    We should choose $a$ places to locate 1's out of $a + b$ places, which is $a + b \choose a$.
\end{proof}

The triangular array of number in which row $n$ consists of all the binomial coefficients with $n$ on top (starting with row 0) is called \textsf{Pascal's triagle}.
\begin{lem} [Pascal's Formula] \label{lem:pascal}
    If $n \geq 1$, then
    \[
        {n \choose k} = {n - 1 \choose k} + {n - 1 \choose k - 1}.
    \]
\end{lem}
\begin{proof}
    Consider a nonempty set $S$ with $n$ elements.
    Suppose we choose $k$ elements from $S$ to construct a subset $T$.
    Choose any element $a \in S$.
    Then either $a \in T$ or $a \notin T$.
    The two cases are disjoint.
    For the case $a \in T$, we need to choose other $k - 1$ elements from $S \backslash \{a\}$, which has $n - 1 \choose k - 1$ cases.
    For the case $a \notin T$, we need to choose $k$ elements from $S \backslash \{a\}$, which has $n - 1 \choose k$ cases.
    From the rule of sum, the total number of cases is ${n - 1 \choose k - 1} + {n - 1 \choose k}$.
    A simpler way of counting $T$'s is just to choose $k$ elements from $n$, which is $n \choose k$.
    Therefore, ${n \choose k} = {n - 1 \choose k} + {n - 1 \choose k - 1}$.
\end{proof}

\begin{thm} 
    With repetition allowed, there are $n + k - 1 \choose k - 1$ ways to select $n$ objects from $k$ types.
    This is also equal to the number of nonnegative integer solutions to $x_1 + \dots + x_k = n$.
\end{thm}
\begin{proof}
    We first show that choosing $n$ objects from $k$ types is equivalent to the number of nonnegative integer solutions to $x_1 + \dots + x_k = n$.
    Let $x_i$ be the number of objects chosen from the $i$-th type.
    Since total $n$ objects are chosen, $\sum_{i = 0}^k x_i = n$.
    Thus, we aim to show that the number of nonnegative integer solutions is $n + k - 1 \choose k - 1$.

    The problem of finding the number of nonnegative integer solutions of $\sum_{i=0}^k x_i = n$ corresponds to arranging $n$ $\square$'s and $k - 1$ /'s:
    \[
        \underbrace{\square \square \dots \square}_n \underbrace{/ / \dots /}_{k-1},
    \]
    where the number of $\square$'s in the $i$-th spacing created by /'s corresponds to $x_i$.
    The number of cases is to place $k - 1$ /'s in $n + k - 1$ positions: $n + k - 1 \choose k - 1$.
\end{proof}

\begin{prob} [Nonnegative Integer Solutions]
    Suppose that each resident of New York City has 100 coins in a jar.
    The coins come in five types.
    We consider two jars of coins to be equivalent if they have same number of coins of each type.
    Is it possible that no two people have equivalent jars of coins?
\end{prob}
\begin{sol}
    Let the number of each type of coin is $x_i$.
    Then, the number of possible combinations of the coins is equal to the number of nonnegative integer solutions of $\sum_{i=1}^5 x_i = 100$.
    The number of the solutions is ${100 + 5 - 1 \choose 5 - 1} = {104 \choose 4} = 4,598,126$.
    The population of NYC is 8.41 million as of 2013.
    From the pigeon hole principle, there are at least two people with the equivalent jars--though the principle is yet to be introduced.
\end{sol}

\begin{cor}
    The expansion of $\left( \sum_{i=1}^m x_i \right)^d$ has $d + m - 1 \choose m - 1$ terms.
\end{cor}
\begin{proof}
    Possible form of terms from the expansion of $\left( \sum_{i=1}^m x_i \right)^d$ is $x_1^{\alpha_1} x_2^{\alpha_2} \dots x_m^{\alpha_m}$, where $\sum_{i=1}^m \alpha_i = d$, and each $\alpha_i$ is a nonnegative integer.
    Thus, it has $d + m - 1 \choose m - 1$ terms.
\end{proof}

\begin{lem} [The Chairperson Identity]
    \[
        k {n \choose k} = n {n - 1 \choose k - 1}
    \]
\end{lem}
\begin{proof}
    Let $S$ be a set with $n$ elements.
    We shall choose a subset $T$ of $S$ and again choose an arbitrary element from $T$, where $|T| = k$.
    Then, choosing $T$ has $n \choose k$ number of cases, and choose one element from it has $k$ number of cases.
    Thus, the overall number of cases is $k {n \choose k}$.

    We can construct $T$ and the element from it in a slightly different manner: first choose any element from $S$ and then choose $k - 1$ elements from $S$ to construct $T$.
    Choosing any element $e$ from $S$ has $n$ number of cases, and choosing $k - 1$ elements from $S \backslash \{e\}$ has $n - 1 \choose k - 1$ number of cases.
    Thus, the number of cases of the whole procedure is $n {n - 1 \choose k - 1}$.
    Therefore, we have the desired relation.
\end{proof}

\begin{thm} [The Summation Identity] \label{thm:bin_sum}
    \[
        \sum_{i=0}^n {i \choose k} = {n + 1 \choose k + 1}
    \]
\end{thm}
\begin{proof}
    We use induction on $n$.
    Consider the base case $n = 0$.
    Then the left hand side becomes
    \[
        \sum_{i=0}^n {i \choose k} = {0 \choose k} = \begin{cases}
            1 &\text{if } k = 0\\
            0 &\text{otherwise}
        \end{cases},
    \]
    whereas the right hand side becomes
    \[
        {1 \choose k + 1} = \begin{cases}
            1 &\text{if } k = 0\\
            0 &\text{otherwise}
        \end{cases}.
    \]
    Thus the given relationship holds for the base case.

    Let us suppose the given relationship holds for some $n = l \geq 1$.
    Then, 
    \begin{align*}
        \sum_{i=0}^{l+1} {i \choose k} &= \sum_{i=0}^l {i \choose k} + {l+1 \choose k}\\
                                       &= {l + 1 \choose k + 1} + {l + 1 \choose k} &\text{by the induction hyp.}\\
                                       &= {l + 2 \choose k + 1} &\text{by Lemma~\ref{lem:pascal}}
    \end{align*}
    Therefore, from the induction principle, the given relationship holds for all $n \in \mathbb{N} \cup \{0\}$.
\end{proof}

\begin{ex} [Summation of Integer Power]
    Formulas for sums of powers are easy to verify by induction but difficult to gues.
    Theorem~\ref{thm:bin_sum} provides a method that automatically generates both the answer and the proof.
    Notice that $i = {i \choose 1}$.
    Therefore, Theorem~\ref{thm:bin_sum} proves the summation formula for the first $n$ natural numbers by 
    \[
        \sum_{i=0}^n i = \sum_{i=0}^n {i \choose 1} = {n + 1 \choose 2} = \frac{n(n+1)}{2}.
    \]
    For the squares, we rewrite $i^2$ using binomial coefficients.
    Since $i^2 = 2 {i \choose 2} + {i \choose 1}$, we have
    \begin{align*}
        \sum_{i=0}^n i^2 &= 2 \sum_{i=0}^n {i \choose 2} + \sum_{i=0}^n {i \choose 1}\\
                         &= 2 {n+1 \choose 3} + {n+1 \choose 2}\\
                         &= \frac{n(n+1)(2n+1)}{6}.
    \end{align*}
    This approach yields $\sum_{k=0}^n f(k)$ for any polynomial $f$.
\end{ex}

\begin{thm}
    For $k \in \mathbb N$, the value of $\sum_{i=1}^n i^k$ is a polynomial in $n$ with the leading term $\frac{1}{k+1} n^{k+1}$ and the next term $\frac 12 n^k$.
\end{thm}
\begin{proof}
    $i^k$ can be written as:
    \[
        i^k = a_k {i \choose k} + a_{k-1} {i \choose k - 1} + \dots + a_1 {i \choose 1}.
    \]
    Thus we have
    \begin{align*}
        \sum_{i=0}^n i^k &= a_k \sum_{i=1}^n {n + 1 \choose k + 1} + a_{k - 1} \sum_{i=1}^n {n + 1 \choose k} + \dots + a_1 \sum_{i=1}^n {n + 1 \choose 1}\\
                         &= a_k {n + 1 \choose k + 1} + a_{k-1} {n + 1 \choose k} + \dots + a_1 {n +  1 \choose 2}\\
                         &= a_k O(n^{k+1}) + a_{k-1} O(n^k) + O(n^{k-1}).
    \end{align*}
    To find the coefficient of the $n^{k+1}$ and $n^k$, we only need to consider $a_k$ and $a_{k-1}$.
    Before getting the exact form of $a_k$ and $a_{k+1}$, we find the coefficients of $n^{k+1}$ and $n^k$ about $a_k$ and $a_{k+1}$.
    \begin{align*}
        a_k {n + 1 \choose k + 1} &= a_k \frac{(n + 1)!}{(k + 1)! (n - k)!}\\
                                  &= a_k \frac{(n + 1) n \dots (n - (k - 1))}{(k + 1)!}\\
                                  &= \frac{a_k}{(k + 1)!} \left(n^{k + 1} + (1 - 1 - 2 - \dots - (k-1)) n^k + O(n^{k - 1})\right)\\
                                  &= \frac{a_k}{(k + 1)!} \left(n^{k + 1} + \left(1 - \frac{k(k-1)}{2}\right) n^k + O(n^{k - 1})\right)\\
                                  &= \frac{a_k}{(k + 1)!} n^{k + 1} + \frac{a_k}{(k+1)!}\frac{-k^2 + k + 2}{2} n^k + O(n^{k-1})\\
    a_{k-1} {n + 1 \choose k} &= a_{k - 1} \frac{(n + 1)!}{k! (n + 1 - k)!}\\
                              &= a_{k-1} \frac{(n+1)n \dots (n - (k-2))}{k!}\\
                              &= \frac{a_{k-1}}{k!} n^k + O(n^{k-1})\\
    \end{align*}
    Thus, we have
    \[
        \sum_{i = 0}^n i^k = \frac{a_k}{(k + 1)!} n^{k + 1} + \left( \frac{a_k}{(k + 1)!} \frac{- k^2 + k + 2}{2} + \frac{a_{k-1}}{k!}\right) n^k + O(n^{k-1}).
    \]

    To get $a_k$ and $a_{k - 1}$, we expand $a_k {i \choose k} + a_{k-1} {i \choose k - 1} + O(n^{k-1})$:
    \begin{align*}
        a_k {i \choose k} + a_{k-1} {i \choose k - 1} &= a_k \frac{i(i - 1) \dots (i - (k - 1))}{k!} + a_{k - 1} \frac{i(i - 1) \dots (i - (k - 2))}{(k - 1)!}\\
                                                      &= a_k \frac{i^k - \frac{k (k-1)}{2} i^{k-1} + O(i^{k-2})}{k!} + a_{k-1} \frac{i^{k-1} + O(i^{k-2})}{(k - 1)!}\\
                                                      &= \frac{a_k}{k!} i^k + \left( \frac{a_{k-1}}{(k-1)!} - \frac{k(k-1)}{2}\frac{a_k}{k!} \right) i^{k-1} + O(i^{k-2})
    \end{align*}
    Therefore, we have $a_k = k!$ and $a_{k-1} = (k-1)! \cdot \frac{k(k-1)}{2}$.
    Substitute these back to $a_k {n + 1 \choose k + 1}$ and $a_{k-1} {n + 1 \choose k}$:
    \begin{align*}
        \sum_{i = 0}^n i^k &= \frac{a_k}{(k + 1)!} n^{k + 1} + \left( \frac{a_k}{(k + 1)!} \frac{- k^2 + k + 2}{2} + \frac{a_{k-1}}{k!}\right) n^k + O(n^{k-1})\\
                           &= \frac{k!}{(k + 1)!} n^{k + 1} + \left( \frac{k!}{(k + 1)!} \frac{- k^2 + k + 2}{2} + \frac{(k-1)! \cdot \frac{k(k-1)}{2}}{k!}\right) n^k + O(n^{k-1})\\
                           &= \frac{1}{k+1} n^{k + 1} + \left( \frac{-k^2 + k + 2}{2(k + 1)} + \frac{k(k-1)}{2k}\right) n^k + O(n^{k-1})\\
                           &= \frac{1}{k+1} n^{k + 1} + \left( \frac{-k^2 + k + 2}{2(k + 1)} + \frac{k^2-1}{2(k + 1)}\right) n^k + O(n^{k-1})\\
                           &= \frac{1}{k+1} n^{k + 1} + \frac 12 n^k + O(n^{k-1}).
    \end{align*}
    We have the desired result.

\end{proof}
\end{document}
